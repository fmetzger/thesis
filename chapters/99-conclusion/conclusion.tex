%!TEX root = ../../dissertation.tex
%%%%%%%%%%%%%%%%%%%%%%%%%%%%%%%%%%%%%%%%%%%%%%%%%%%%%%%%%%%%%%%%%%%%%%%%%%%%%%%%
\chapter{Conclusions}
\label{chap:conclusion}

Reliable streaming deviates from the established ways in a number of significant ways. In the past, streaming was dominated by server-controlled push-based approaches. Specialized and highly complex software --- both proprietary, in the form of Flash and other players, and standards-based using \gls{rtp} --- was needed to properly conduct streaming. 

But many video streaming solutions deployed in the past few years have shifted to using simple client-side players better suited for the Web's ecosystem. Today, any Web browser can pull video data and display it directly inside a video element without the help of any additional software. The popularity of reliable video streaming services has increased in such a way, that they are now one of the largest source of the Internet's traffic and are predicted to rise even further. But all these reliable streaming --- so-called due to them operating on top of the reliable \gls{TCP} transport protocol --- players have in common that modeling, optimization, and measurement techniques used for unreliable streaming can not be applied to them any more due to their different approaches.


A second development in recent years is the increasing prevalence of cellular mobile networks, taking the form of \gls{UMTS} and \gls{LTE}. Smartphones have become omnipresent and traffic originating from these devices makes up an increasing portion of the Internet's total traffic.



Again make the Internet stack vs. mobile stack comparison

mention scalability issues of control plane in general and that there is no actual need for this

compare on basis of core networking principles: KISS, E2E, layering/encapsulation, 





\section{Future Work}
