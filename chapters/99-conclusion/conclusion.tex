%!TEX root = ../../dissertation.tex
%%%%%%%%%%%%%%%%%%%%%%%%%%%%%%%%%%%%%%%%%%%%%%%%%%%%%%%%%%%%%%%%%%%%%%%%%%%%%%%%
\chapter{Conclusions}
\label{chap:conclusion}

Reliable streaming deviates from the established ways in a number of significant ways. In the past, streaming was dominated by server-controlled push-based approaches. Specialized and highly complex software --- both proprietary, in the form of Flash and other players, and standards-based using \gls{rtp} --- was needed to properly conduct streaming. 

But many video streaming solutions deployed in the past few years have shifted to using simple client-side players better suited for the Web's ecosystem. Today, any Web browser can pull video data and display it directly inside a video element without the help of any additional software. The popularity of reliable video streaming services has increased in such a way, that they are now one of the largest source of the Internet's traffic and are predicted to rise even further. But all these reliable streaming --- so-called due to them operating on top of the reliable \gls{TCP} transport protocol --- players have in common that modeling, optimization, and measurement techniques used for unreliable streaming can not be applied to them any more due to their different approaches.

A second development in recent years is the increasing prevalence of cellular mobile networks, taking the form of \gls{UMTS} and \gls{LTE}. Smartphones have become omnipresent and traffic originating from these devices makes up an increasing portion of the Internet's total traffic. While the user plane traffic is still rather low when compared to wired Internet access, mobile networks have a much more intricate control plane and any user plane traffic causes a number of related control plane interactions which were discussed in \ref{c4:sec:3gpparchitecture}. The actual volume of these signaling messages might be low, but they cause secondary load effects in the nodes that need to process the control plane data and store its state. And these effects are barely investigated or even factored into any network dimensioning and planning tasks.

When work on this dissertation thesis had started, both of these fields began to grow together. Smartphones developed to a state where watching video streams on their displays was not a hassle any more and there was sufficient capacity in the mobile access to transport video streams of relatively high quality in a timely fashion.

But again both also seemed to lack proper measurement and evaluation methods and models. This is what was tackled in this work: 

One one side, a categorization approach and measurement model for reliable progressive video streaming, that bases itself on emulating the streaming client's video playback and strategies, was introduced. With this at hand, a number of different strategies, cloned from previously existing real streaming solutions, were tested in a network emulation testbed and found an interaction of the connection's latency and loss properties with the video streaming stalling characteristics. This measurement model can be utilized to test new playback strategies for real applications.

On the other side, light was shed on the control plane interactions of a typical \gls{3G} core network. The mobile networks' overly strong focus on signaling procedures and statefulness can be the source of numerous scalability issues as was previously explored for the radio interface. Here, through the analysis of network traces of a large operational mobile core network a novel definition of load in the core network was created. This load was put in relation to \gls{gtp} tunnel management signaling between the two central core nodes \gls{SGSN} and \gls{GGSN}. 

Diurnal variations and dependencies on the user's device and its user traffic characteristics could be observed in the trace. A queuing theory model that describes the load at a \gls{GGSN} the load properties was derived from these findings and takes the form of a non-stationary Erlang loss model. With the help of this model a new spin on the \gls{GGSN}'s makeup using virtualization was introduced and tested in simulations to provide better load scaling options.

Testing reliable video streaming in a mobile network and understanding its apparent behavior might be a very interesting combination of the two seemingly independent fields. But it also became clear during the evaluation of the mobile dataset, that specific user traffic patterns can have a direct influence on the core's control plane. Segmented reliable streaming can show a certain ON-OFF style traffic pattern, that may interact badly with the timers governing the control plane state and increase the core load.

This kind of negative interaction could neither be verified nor refuted during the course of this thesis due to the lack of data. However, valuable approaches on how to test for this were still shown in the form of active measurements enhanced with smartphone metadata and a streaming simulation framework on top of a ns-3-based \gls{LTE} network.

Once issues or non-optimal situations have been discovered through these means, they can be mitigated and improved upon. For example, this can happen by adapting the playback strategies and protocols that are already being utilized --- \gls{TCP} is the best example of being constantly modified --- by employing new protocols, such as \gls{DASH} or \gls{HTTP}/2 or utilizing cross-layer information to its fullest.

And these are also the lessons learned during the course of this thesis's work. Understanding and optimizing a certain item begins by investigating, evaluating and modeling existing iterations of it and mapping out the limits and possibilities. As reliable streaming in a mobile network is a relatively new application displaying very complex interactions and was generally not very well investigated, a lot of groundwork had to be done. And the measurement models provided here attempt to do just that. Future research can utilize the work conducted here as a foundation. It would be great to see, if the work spawns a number of scientific offspring that try to answer some of the questions and issues raised here. Concerning the influences on the performance of mobile streaming. Even without further studies, a lasting benefit may already be present for both video streaming implementers and mobile network operators. They have now access to more tools to optimize their products for specific scenarios.

It was also shown, that optimization in general does not automatically constitute making networks and applications more complex or putting more intelligence into the network. On the contrary, it is often better to uphold some core principles that have driven computer networking and the Internet thus far. This includes \gls{KISS}, the focus on end-to-end systems, and also keeping a layered network stack. 

Many of the issues with streaming in mobile networks discussed in this work have arisen by not adhering to these principles. The \gls{UMTS} control plane is so tremendous, spanning many network nodes and merging many protocol layers into one big specification, that many undesirable interactions occur. The original developers could not have foreseen all future use cases and what side effects they might bright with them. On the flip side, this thesis would not even have been possible without the many peculiarities found in these \gls{3G} specifications.


%%
\section{Future Work}

Alas, work on all these individual items discussed here is certainly not done. As usual in scientific work, progress will be made in small iterative steps. Therefore, this final section collects some planned work and ideas that follow up on the thesis's research angles in the future.

Concerning the modeling and measurement of reliable streaming it is important to keep up with the current flavors of playback strategies. Now that adaptive streaming is being used more widespread adaptive strategies should also be taken more into consideration. In addition, the changes on other protocol layers have to be constantly monitored for their potential influence on streaming. New protocols should be tested if they are viable for reliable streaming. 

For example, \gls{HTTP}/2's improved multiplexing features could be worthwhile for the simultaneous transport of multiple streaming segments. This is especially helpful when \gls{SVC} is employed to provide the various quality levels in adaptive streaming. Multiplexing would allow multiple layers to be transported with differing priority simultaneously using just one connection.

The more strategies are known and understood the more candidates are available in the search for the best strategy fitting a given scenario or applicable to most cases. A good candidate here is probably an adaptive threshold strategy which observes its buffer fill rate in a sliding window.

When the base effects of the stalling characteristics are well understood, more complex metrics could also be derived from this to better represent the \gls{QoE}. A potential objective \gls{MOS} computation could relate these basic statistics to subjective quality impressions gained from user studies.

%% 
Moving on to mobile core networks, most of the deductions were made on the basis of a single core network trace from one network and need to be verified with additional and more recent traces from other networks. This could simultaneously observe if there have happened any changes in the time after the original trace was recorded.

That trace did also lack some particular interesting details regarding the actual \gls{gtp} \gls{IE} data from the messages and control plane procedures on off-user-traffic-path links (e.g., to the \gls{HLR}). Any subsequent recording should therefore factor in more information and should provide measurements at several synchronized vantage points inside the network to better correlate the individual message. Additional probes at the radio or backhaul links could prove especially fruitful in this endeavor. 

But as the usage \gls{UMTS} is declining in favor of \gls{LTE} or even \gls{LTE}-Advanced, the core control plane in these networks also needs to be put under close scrutiny, and not just in a simulator as in this work. A multi-vantage point measurement campaign in an actual \gls{EPC} can make an informed comparison to the existing \gls{3G} trace and find improvements or regressions. As the combination of \gls{SGSN} and \gls{GGSN} is replaced by \gls{SGW}/\gls{PGW} with a slightly different subset of functionalities a new core load queuing model should also be created in the course of the trace's statistical evaluation with these differences in mind. This can also lead to better load models once more core network elements --- especially the off-user-traffic path elements, including the \gls{HSS}, \gls{MME}, or \gls{PCRF} --- are evaluated and factored into the queuing model.

Through this process, suggestions for future further iterations of the mobile network control plane specifications can be derived and taken into consideration. Any upcoming measurement campaign should also consider running a concurrent reliable video streaming active measurement experiment in the observed network. A correlation between streaming traffic and occurring control plane signaling could be drawn much more easily.


%%
To better observe these interactions between streaming and mobile networks a lot of work still has to be done to bring mobile network simulators up to parity with actual networks. A better control plane representation in, for example, ns-3 or OMNeT++ would go a long way in increasing the validity of the results derived from these simulations. If a new, more fitting simulation platform arises, the reliable streaming model can also be reimplemented there as it is kept portable.

Similarly, simulators are also lacking behind in their representation of current higher layer protocols, especially \gls{TCP}. Solutions need to be explored to remedy this and make the simulated stacks more comparable to the fast changing stacks of current \glspl{os}. Small changes here can have a noticeable impact on the application layer and can also falsify simulation results, as they are not applicable to any actual network.

Finally, regarding the ideas for a cross-layer information exchange, the general cross-layer information broker should be implemented on different architectures (e.g., an Android smartphone and a Linux laptop) to test different protocol stacks and different amounts of cross-layer information sources and sensory input. An actual implementation of the handover-aware adaptive playback strategy should also be relatively easy to achieve as a prototype to test the broker in a practical situation.




%impact?
%vision for the future with this work
%Again make the Internet stack vs. mobile stack comparison
%compare on basis of core networking principles: KISS, E2E, layering/encapsulation, 
