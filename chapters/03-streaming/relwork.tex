%!TEX root = ../../dissertation.tex
%%%%%%%%%%%%%%%%%%%%%%%%%%%%%%%%%%%%%%%%%%%%%%%%%%%%%%%%%%%%%%%%%%%%%%%%%%%%%%%%
\section{Related Work}


Video streaming touches many aspects of computer communication network research. The technical fundamentals for video streaming have existed for a sufficiently long time so that there is a large body of existing work. The presented research 
% This section arranges relevant related work topics on a protocol layer top-down approach. First, publications related to video streaming applications and their mechanisms are presented. Afterwards, work on streaming transport is discussed.

The chapter's focus lies on reliable \gls{HTTP} streaming to which \cite{watching-video1,watching-video2} and \cite{ma2011mobile} give an introduction and overview the mechanics involved in streaming, e.g. flow control mechanisms in the video delivery. Akhshabi et al. \cite{akhshabi2011experimental} take a look at real world streaming implementations and conduct comparative measurements. Initial experiments evaluate the viability of this kind of approach. 

Concerning specific streaming solutions, there are several publications discussing YouTube's architecture. In 2007 Gill et al. \cite{gill2007youtube} made a long-term observation of YouTube traffic originating from an university network. Their analysis shows detailed characteristics of the served videos, among others file sizes, durations, and bitrates, and reveals a daily number of video requests. Adhikari et al. \cite{adhikari2010youtube} collected data from points of presence of one ISP to explore the service's traffic distribution and load balancing techniques. However, these were still based on YouTube's old architecture prior to being acquired by Google. The new infrastructure differs largely from the old, e.g. load balancing and content distribution now exclusively uses Google's network. Mori et al. describe in \cite{mori2010characterizing} distinctive attributes of video traffic flows originating from YouTube's new and current setup while Torres et al. \cite{torres2011dissecting} focus on observations of the \gls{CDN}'s server selection process using multi-network passive measurements. Concerning \glspl{CDN}, \cite{Labovitz:2010:IIT:2043164.1851194} showed the importance of this new traffic distribution approach in interdomain traffic. Web and video traffic was also on the rise, both is to be expected through the presence of large video distributing Web sites.


The authors of \cite{wang2003model} propose an analytical model for TCP-based video streaming, differentiating between live and pre-recorded videos. The impact of packet loss on an unreliable video stream is studied in \cite{pv2010loss}. The loss-hiding properties of reliable streaming makes this study only somewhat applicable to \gls{HTTP} streaming. In \cite{pv2010qoe}, the authors present a quality-assessment model for video streaming services, with the quality features derived from the actual video. The model does not include the network behavior, but focuses on the codec performance instead. 

To determine the quality of the streaming process as well as any models resembling the process, metrics have to be used or the so-called \gls{QoE}, either subjectively or objectively, measured in some way. While an overview will be given in Section~\ref{c3:sec:metrics}, a few selected publications are already presented here. A metric termed ``application comfort'' defined and applied to YouTube videos in \cite{staehle2010yomo} to monitor live network conditions in realtime. While this approach is in effect similar to our evaluations, it is geared towards a very specific implementation of streaming, whereas our presented methodology is more generic. YouTube's \gls{QoE} is not just up to the sender and receiver. Some control could also be placed in an access network, manipulating YouTube streams to improve streaming quality. This was conducted in \cite{5733220} for a wireless mesh network.
In 2012, a publication \cite{6296879} presented approaches to derive YouTube's playback buffer and quality from passive measurements inside the network. Approaches like this can be used by \gls{ISP} to check their network and estimate quality customers are achieving. A publication by Hoßfeld et al. \cite{6123395} identifies \gls{QoE} influencing factors on YouTube streaming through subjective \gls{MOS} collected by a crowdsourcing method. The number of stalling events is seen as the factor with the highest impact. This idea is furthered by \cite{6263849} with a comparison between the initial day of a stream's start and the number of interruptions, with stalling resulting in a much lower \gls{MOS}.

Observations performed in \cite{ketyko2010qoe} on the Android platform were showing that re-buffering events, i.e. phases of video stalling, can result in a large drop in \gls{MOS}. The authors of \cite{mokmeasuring} measure \gls{QoE} effects of \gls{HTTP} video streaming in a controlled test setup and also conclude that degraded network \gls{QoS} increases re-buffering frequency and decreases the \gls{MOS}. Gustafsson et al. \cite{gustafsson2008measuring} investigate the loss in perceived streaming quality and establish a parametric objective opinion model.
\cite{6181070} presents another \gls{QoE} model that attempts to estimate the quality of adaptive streaming with neural network trained through subjective tests.

Adaptive streaming has also been a topic of intense research. \cite{DeCicco:2011:FCA:1943552.1943573} investigates quality adaptation techniques and proposes a feedback mechanism for quality control. Unfortunately, this places control solely at the server side, contrary to the current trend of the streaming client exercising full control. The authors of \cite{cicco2010akamai} conducted measurements of yet another server-controlled adaptive streaming mechanism, which is employed by Akamai's video streaming.Moreover, according to \cite{5945211}, the circumstance, that \gls{TCP} throughput does not automatically throttle itself to the current video bitrate, could be problematic. The paper proposes and tests an additional scaling mechanism in a simulation.

To get a grip on the real-world behavior of streaming mechanisms, many measurement studies are conducted. In measurements, especially in passive measurements, one typically cannot measure the application protocol on its own. Rather, the whole network stack, starting from \gls{IP} packets and upwards is captured and must be evaluated for the specific property under investigation.
In \cite{Erman:2011:OTV:2068816.2068829} such a general traffic study is conducted in a cellular network, estimating the ubiquity of video streaming, even in mobile networks. The authors of \cite{Huang:2012:CTU:2398776.2398800} conclude in their proxy-based active measurements that many adaptive streaming approaches underutilize the available network bandwidth and achieve a lower quality than they could. An analytical ON-OFF model is developed in \cite{Rao:2011:NCV:2079296.2079321} through the evaluation of active measurements comparing different streaming strategies.
In a survey of several streaming protocols \cite{5703713} the overhead on the transmission of each of them is investigated and compared based on an analytical approach. 


	%\item HTTP Live Streaming \cite{pantos2011livestreaming}
	%\item FoG and Clouds: Optimizing QoE for YouTube \cite{hossfeld2011fog}
	%\item Adaptive streaming: The network HAS to help \cite{BLTJ:BLTJ20505}
	%\item Google now second-largest ISP, carries 6.4\% of Internet traffic \cite{nw2010carrier}
	%\item ComScore February 2011 U.S. Online Video Rankings \cite{comscore2011ranking}
	%\item WebRTC / Overview: Real Time Protocols for Brower-based Applications\cite{ietf2011rtcwebdraft}
	%\item Quantification of quality of experience for edge-based applications\cite{hossfeld2007quantification} (nur für skype)
	%Cloud gaming as a media streaming application with strong real-time requirements \cite{4795441,wang2009modeling,jarschel2011cloudevaluation,ct2010wolken}
%IIS smooth streaming technical overview \cite{zambelli_iis_2009}
	%\item What DNS is not. \cite{vixie2009dns}
	%\item Comparing DNS resolvers in the wild \cite{ager2010comparing}
	%\item Drafting behind Akamai (travelocity-based detouring) \cite{Su:2006:DBA:1159913.1159962}