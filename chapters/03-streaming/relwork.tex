%!TEX root = ../../dissertation.tex
%%%%%%%%%%%%%%%%%%%%%%%%%%%%%%%%%%%%%%%%%%%%%%%%%%%%%%%%%%%%%%%%%%%%%%%%%%%%%%%%
\section{Related Work}

Video streaming touches many aspects of computer communication network research. The technical fundamentals for video streaming have existed for a sufficiently long time so that there is a large body of existing work. This section attempts to summarize some of the more closely related research approaches with today's reliable streaming mechanisms as a central theme.

The chapter's focus lies on reliable \gls{HTTP} streaming to which \cite{watching-video1,watching-video2} and \cite{ma2011mobile} gave an introduction and overview the mechanics involved in streaming, e.g., flow control mechanisms in the video delivery. Akhshabi et al.~\cite{akhshabi2011experimental} took a look at real world streaming implementations and conducted comparative measurements. Initial experiments evaluated the viability of this kind of approach. 

Concerning specific streaming solutions, there are several publications discussing YouTube's architecture. In 2007 Gill et al.~\cite{gill2007youtube} made a long-term observation of YouTube traffic originating from an university network. Their analysis showed detailed characteristics of the served videos, amongst others file sizes, durations, and bitrates, and revealed a daily number of video requests. Adhikari et al.~\cite{adhikari2010youtube} collected data from points of presence of one \gls{ISP} to explore the service's traffic distribution and load balancing techniques. However, these were still based on YouTube's old architecture prior to being acquired by Google. The new infrastructure exhibited large differences to the old one. For example, load balancing and content distribution now exclusively use Google's network and is no longer based on Akamai's infrastructure. 

Mori et al.\ described in \cite{mori2010characterizing} distinctive attributes of video traffic flows originating from YouTube's new and current setup, while Torres et al.~\cite{torres2011dissecting} focused on observations of the \gls{CDN}'s server selection process using multi-network passive measurements. Concerning \glspl{CDN}, \cite{Labovitz:2010:IIT:2043164.1851194} showed the importance of this new traffic distribution approach in an interdomain traffic study. The amount of Web and video traffic was seen to be on the rise, both is to be expected through the presence of large video distribution Web sites.


%%
The authors of~\cite{wang2003model} proposed an analytical model for \gls{TCP}-based video streaming, differentiating between live and pre-recorded videos. The impact of packet loss on an unreliable video stream is studied in~\cite{pv2010loss}. However, the loss-hiding properties of reliable streaming makes this study only somewhat applicable to \gls{HTTP} streaming. In~\cite{pv2010qoe}, the authors presented a quality-assessment model for video streaming services, with the quality features derived from the actual video. The model does not include the network behavior, but it rather focused on the codec performance instead. 

Video quality, or so-called \gls{QoE} metrics, help in determining the quality of the streaming process and of models resembling the process. The metrics can be either subjective or objective and are further discussed in Section~\ref{c3:sec:metrics}. A few select publications are already presented here.

A metric coined ``application comfort'' was defined and applied to YouTube videos in~\cite{staehle2010yomo} to monitor live network conditions in realtime. It is geared towards a very specific implementation of streaming, whereas the measurement methodology presented here is more generic. While YouTube's horizontal location of control is originated at the endpoints, some third-party control unit could also be placed in an access network, manipulating YouTube streams in an attempt to improve streaming quality. This was conducted in~\cite{5733220} for a wireless mesh network.

In 2012, a publication~\cite{6296879} presented approaches to derive YouTube's playback buffer and quality from passive measurements inside the network. Approaches like this can be used by \glspl{ISP} to check their network quality and estimate the quality customers are achieving. 

A publication by Hoßfeld et al.~\cite{6123395} identifies \gls{QoE} influencing factors on YouTube streams through subjective \glspl{MOS} collected by a ``crowdsourcing'' method. The number of stalling events was revealed to be the factor with the highest impact on the \gls{QoE}. This idea was furthered by \cite{6263849} with a comparison between the initial delay of a stream's start and the number of interruptions. Stalling resulted in a much lower \gls{MOS}. Observations performed in \cite{ketyko2010qoe} on the Android platform were also showing that stalling events can result in a large drop in \gls{MOS}. 

The authors of \cite{mokmeasuring} measured \gls{QoE} effects of \gls{HTTP} video streaming in a controlled test setup and conclude that degraded network \gls{QoS} increases stalling frequency and decreases the \gls{MOS}. Gustafsson et al.~\cite{gustafsson2008measuring} investigated the loss in perceived streaming quality and established a parametric objective opinion model. \cite{6181070} presents another \gls{QoE} model that attempts to estimate the quality of adaptive streaming with a neural network trained through subjective tests.

Adaptive streaming has also been a topic of intense research. \cite{DeCicco:2011:FCA:1943552.1943573} investigated quality adaptation techniques and proposes a feedback mechanism for quality control. Curiously this places control solely at the server side, contrary to the current trend of the streaming client exercising full control. The authors of \cite{cicco2010akamai} conducted measurements of yet another server-controlled adaptive streaming mechanism, which is employed by Akamai's video streaming. Moreover, according to \cite{5945211}, the circumstance, that \gls{TCP} throughput does not automatically throttle itself to the current video bitrate, could be problematic. The paper proposes and tests an additional scaling mechanism in a simulation.

To get a grip on the real-world behavior of streaming mechanisms, many measurement studies are conducted. In measurements, especially in passive measurements, one typically cannot measure the application protocol on its own. Rather, the whole network stack, starting from \gls{IP} packets and upwards is captured and must be evaluated for the specific property under investigation. In \cite{Erman:2011:OTV:2068816.2068829} such a general traffic study is conducted in a cellular network and found video streaming traffic as ubiquitous as in wired networks. The authors of \cite{Huang:2012:CTU:2398776.2398800} concluded in their proxy-based active measurements that many adaptive streaming approaches underutilize the available network bandwidth and achieve a lower quality than they could. An analytical ON-OFF model was developed in \cite{Rao:2011:NCV:2079296.2079321} through the evaluation of active measurements comparing different streaming strategies. And finally, in a survey of several streaming protocols in \cite{5703713} the overhead on the transmission of each of them was investigated and compared based on an analytical approach. 


	%\item HTTP Live Streaming \cite{pantos2011livestreaming}
	%\item FoG and Clouds: Optimizing QoE for YouTube \cite{hossfeld2011fog}
	%\item Adaptive streaming: The network HAS to help \cite{BLTJ:BLTJ20505}
	%\item Google now second-largest ISP, carries 6.4\% of Internet traffic \cite{nw2010carrier}
	%\item ComScore February 2011 U.S. Online Video Rankings \cite{comscore2011ranking}
	%\item WebRTC / Overview: Real Time Protocols for Brower-based Applications\cite{ietf2011rtcwebdraft}
	%\item Quantification of quality of experience for edge-based applications\cite{hossfeld2007quantification} (nur für skype)
	%Cloud gaming as a media streaming application with strong real-time requirements \cite{4795441,wang2009modeling,jarschel2011cloudevaluation,ct2010wolken}
%IIS smooth streaming technical overview \cite{zambelli_iis_2009}
	%\item What DNS is not. \cite{vixie2009dns}
	%\item Comparing DNS resolvers in the wild \cite{ager2010comparing}
	%\item Drafting behind Akamai (travelocity-based detouring) \cite{Su:2006:DBA:1159913.1159962}