%!TEX root = ../../dissertation.tex
%%%%%%%%%%%%%%%%%%%%%%%%%%%%%%%%%%%%%%%%%%%%%%%%%%%%%%%%%%%%%%%%%%%%%%%%%%%%%%%%
\section{Streaming Modeling}
\label{c3:sec:modeling}

As stated, the goal of this chapter is to model the measuring process and reliable streaming. This section will introduce the proposed reliable streaming measurement model and with it some reference playback strategies that cover many of the potential aspects of streaming. It is intended for an easy comparison between different protocol variants and to evaluate all strategies in one framework. Because any kind of evaluation requires metrics, and reliable streaming has special requirements in this regard, appropriate metrics are researched first.


%%%%%%%%%%%%%%%%%%%%%%%%%%%%%%%%%%%%%%%%%%%%%%%%%%%%%%%%%%%%%%%%%%%%%%%%%%%%%%%%
\subsection{Metrics for Reliable Transport Streaming}
\label{c3:sec:metrics}

When measuring anything related to video or even just image quality, one has the choice between conducting a subjective or objective assessment. 

During a subjective test, human assessors evaluate and rate video quality in a controlled environment with the results usually being aggregated into an overall relative quality score denominated \gls{MOS}. Through the human element, conducting a subjective assessment is very time and resource consuming but it can also serve as the best indicator of the \gls{QoE}.

This is where, objective video quality assessments come into play. Modern objective models attempt to recreate the features of the humans' visual perception and psychological model and are usually calibrated by subjective assessments. Most objective models operate on a full reference approach, meaning that they directly compare the original reference video to the resulting video after being encoded or transmitted.

Two of the simplest full reference image quality metrics, which can also be applied on video, are the \gls{MSE} and \gls{PSNR}, defined as:


\begin{align}
MSE &= \frac{1}{N} \sum_{i=1}^{N}(x_i - y_i)^2\\
\shortintertext{and}
PSNR &= 10 \log_{10} \frac{L^2}{MSE},
\end{align}


where $N$ denotes the number of pixels in a frame and $x$ and $y$ the individual pixels from the reference and output frame respectively. $L$ denotes the maximum value of a pixel. Usually $L = \SI{8}{\bit} = 255$ as the image is either grayscale or, if it has color information, each color channel is calculated separately. \cite{objective-vqa} offers more information on these metrics.

Image quality models can by nature only test for spatial distortions of single images. This includes artifacts like general blockiness or blurriness, noise, or reduced resolution. Quality assessments specifically dedicated to video material can additionally take temporal metrics into account, for examples anomalies in the frame rate. Such models are also being researched and standardized by the \gls{ITU} and \gls{VQEG} for example in \cite{ituJ144, ituJ246, ituJ247}. Often, a network's \gls{QoS} parameters can also be directly mapped onto an objective \gls{QoE} metric as described in \cite{5430142}.

metrics dedicated solely to streaming quality measurements should restrict themselves to measure only degradations that occur during the streaming process and not the initial encoding process. Only a specific subset of quality alterations apply here. For example, lost or late packets can cause missing blocks in a frame or frames to be skipped completely. 

Initial thoughts concerning \gls{IPTV} --- which represents an unreliable streaming solution --- \gls{QoE} have been given in \cite{ituG1080} and the influence of packet delay variations on playback buffers is investigated in \cite{rfc3393}. The \gls{MDI}~\cite{rfc4445} is an attempt to capture this behavior and relate it to the network \gls{QoS}. Its metric relies on two properties, the \gls{DF}, as a measure of the network's latency and jitter, and the media loss rate. Of special interest to this investigation is the \gls{DF}, which is calculated based on a \gls{VB} of received stream data as


\begin{align}
	VB &= r_{rcv} - r_{drain} \\
	DF_i &= \frac{\max(VB) - \min(VB)}{r_{drain}}
\end{align}

with $r_{rcv}$ and $r_{drain}$ denoting the buffer's receiving and draining rates.

Reliable streaming has even less possibilities to degrade a video stream. Packets can not arrive out of order at the application and loss is fully concealed by \gls{TCP}. This means that the transmitted and the played video are always identical and no image reference models are necessary here. The only thing that can still happen, is that portions of the video data arrive too late to be played out at their intended point in time. Reliable streaming quality assessment metrics needs to keep track of the following properties:

\begin{itemize}
	\item The initial delay, which is the time delta between the start of the transmission and the start of the video play.

	\item The number and lengths of interruptions or stalls during playback. More complex metrics could also keep track of the \acrshort{IAT} of stalling events.

	\item For adaptive streaming, the characteristics of the quality levels the video was played in. This includes the frequency of quality switching events and the duration of each level.
\end{itemize}

Few attempts for concise reliable streaming metrics that cover all mentioned properties have been made to date. The \gls{CI} was defined in \cite{1498486} and used to determine quality in \gls{P2P} live streaming. It is defined as \textit{\enquote{the number of segments that arrive before or on playback deadlines over the total number of segments}} and with this partly captures the stalling property. In \cite{5634160} and \cite{DBLP:journals/corr/SeyedebrahimiBP13} a so-called \gls{PI} is defined and evaluated. The definition $I_p = uv$ is simply based on the number of stalls $N$ and the average stall duration $v$. Very similarly, an \gls{ITU} recommendation~\cite{ituP1202.1} defines a rebuffering artifact value as 

\begin{equation}
	\phantom{,}v = 1 - ({\frac{d_s}{d_v + d_s}})^{0.0737}\text{.}
\end{equation}

 with the total stalling duration $d_s$ and the video duration $d_v$. The model lacks to capture other stalling characteristics and quality levels. A final metric by Hoßfeld et al.~\cite{hossfeld2013youtubeqoe} remedies this and incorporates both the lengths $L$ as well as the number $N$ of stalls in the computation of the \gls{MOS}. It states

\begin{equation}
	\phantom{.} f(L,N) = 3.50e^{-(0.15L +0.19)N} + 1.50 \text{ for } L \in \mathbb{R}^{+}, N \in \mathbb{N}.
\label{c3:eqn:hossfeld-stalling-model}
\end{equation}

The equation was derived from user studies on perceived YouTube video quality. Therefore, it can serve as a \gls{QoE} metric for the non-adaptive measurement model introduced in the following section. Regarding adaptive streaming, work conducted in \cite{6603210} gives an overview over some initial attempts of metrics that include temporal aspects of the image quality. But none of them includes all necessary aspects to assess reliable adaptive streaming.

The streaming measurement models presented in the following section derives only the basic properties, including the stalling and quality level characteristics and could be used for any of the above metrics. For non-adaptive reliable streaming it is recommended to use the model defined in Equation~\ref{c3:eqn:hossfeld-stalling-model}. Surpassing this model or defining an entirely new adaptive metric would require extensive user studies and is beyond the scope of this work.

%%%%%%%%%%%%%%%%%%%%%%%%%%%%%%%%%%%%%%%%%%%%%%%%%%%%%%%%%%%%%%%%%%%%%%%%%%%%%%%%
\subsection{Measurement and Playback Model}
\label{c3:sec:model}

Parts of the model presentation were previously published in \cite{cs3518}, \cite{metzger2011delivery}, and \cite{6229739}. It is based on the desire to compare all in-the-wild variants of reliable streaming protocols in a simple and concise way. This is achieved by basing the model on the component that is common to all of the approaches: the playback buffer.

To display a video stream, an application needs to maintain a playback buffer of sufficient size to at least gather enough data to reconstruct the next portion of the video to be displayed. From the perspective of a player application, a video consists of a sequence of atomic units, namely video frames and audio samples, which need to be displayed at certain points in time, predetermined through the video's frame rate. 

The application progressively decodes the video from a source and stores the units temporarily in a memory buffer before playing them. In reliable streaming, the buffer is filled by the payload from received \gls{TCP} segments and subject to the network \gls{QoS}. The process can be subsumed as:

\begin{equation*}
	\mathit{buffer}(t) = \sum_{0}^{t} \text{data}_\mathrm{received} - \sum_{0}^{t} \text{data}_\mathrm{played}
\end{equation*}

Both the incoming and outgoing data stream are variable over time. The fill level of the playback buffer is the critical component in the playback process and the central element of the model. If the buffer reaches a size of zero the playback process must stop and stalling occurs.

\begin{figure}[htb]
	\centering
	\includegraphics[width=0.9\textwidth]{images/playback-model.pdf}
	\caption{Reliable streaming playback model based on buffer control.}
\label{c3:fig:playback-model}
\end{figure}

Figure~\ref{c3:fig:playback-model} overviews the reliable streaming model. The controller, part of the video player, selects a video from a remote location and the transmission is started, filling the playback buffer. The model has four degrees of freedom, which are all governed by the controller and together are coined \textbf{playback strategies}. These are:

\begin{itemize}
	\item The initial playback delay, which is the time between the initiation of the video stream transmission and the actual stream playback. The larger this is chosen, the bigger the safety margin on the buffer gets. If the video and transmission bitrate are known to be constant and and appropriately dimensioned, the initial delay can be chosen to be very small.

	\item Playback pause and resume decisions based on the current buffer fill level. This is a generalization of the initial playback delay, which is in fact only one, albeit always occurring stalling period.

	\item If the video is segmented the segment retrieval rate is also a factor. Individual segments can be requested at a higher rate to fill the buffer more quickly. With this the client can react to changes in network conditions.

	\item Selection of the video or segment quality level with a video bitrate chosen according to the current network throughput. This is only applicable for adaptive streaming of course.
\end{itemize}


These decisions yield a stalling period distribution for a streamed video. The frequency and the duration of stalls directly relate to the decision function of the playback strategy. Under insufficient network conditions the playback strategy has to decide which property is less likely to negatively impact the user. Therefore, either the frequency or the length of stalling events will have to be adjusted. The more frequent the stalls are, the shorter they will be. If the strategy produces longer stalling events, they will be less frequent assuming the same network conditions.

The time scale on which streaming applications buffer content usually lies in the range of seconds. This is a necessity in best-effort networks, as the available network bitrate might drop unexpectedly and cause stalling. To keep the buffer level high, an adaptive streaming client can also reduce the stream's quality level. But it is still an open question if playing at low quality without interruption gives a better \gls{QoE} than high quality with interruptions.

The the rest of this sections present fundamental playback strategies and strategy building blocks with features extracted from real world examples, which are given afterwards. 



%%
\subsubsection{Null Strategy}

The simplest strategy is having no strategy at all. Playback is started immediately when at least a single frame fully resides within the buffer and stops again at an empty buffer. The behavior can be summarized as \textit{``Whenever anything can be played from the buffer, do so.''}.

This results in frequent stops and a large loss in playback continuity and will therefore not be used in practice. However, this strategy has some interesting theoretical properties, which is why it is mentioned here. Both the total stalling time and the required buffer space are minimized. Moreover, the strategy results in an upper limit for the number of stalls occurring\footnote{As a video frame is atomic, no other model could possibly stop the playback more often.}. Therefore, it can act as a baseline reference to assess the performance of other strategies.

\begin{figure}[htb]
	\centering
	\includegraphics[width=0.9\textwidth]{images/R-bufferlevel-stall.pdf}
	\caption{Buffer fill level with null strategy; \SI{33}{\second} total stalling.}
\label{c3:fig:bufferlevel-stall}
\end{figure}

Figure~\ref{c3:fig:bufferlevel-stall} depicts an exemplary time series diagram of the contents of a video buffer using this strategy. The transmission rate was only slightly above the stream's video bit rate. The buffer frequently drops down to zero forcing a short stall. According to the related work in \cite{6123395}, on the \gls{QoE} impact of stalling frequency in comparison to the length of stalls, this is the worst possible scenario for a person watching the stream.


%%
\subsubsection{Threshold Strategies}

Instead of instantly restarting playback a threshold can be introduced. Only after a certain buffer fill level threshold has been surpassed, playback will be started. Thresholds can be set independently for the initial playback delay and stalls, with the initial playback delay generally set to be higher.

The threshold can be chosen in a number of ways. It can either be an absolute data volume or a buffered video duration. The latter is much more suited for variable bitrate videos as it automatically adapts itself to the current bitrate. A third option is to buffer for a certain amount of real time --- this can be seen as threshold --- and to start playback after that period regardless of the volume of the buffer. Additionally, the threshold could also either be set to a constant value or dynamically chosen according to the expected network \gls{QoS}.

Besides this single-threshold strategy two-threshold or multi-threshold strategies might make more sense for segment-based streaming. In addition to the lower threshold, an upper threshold is introduced. When reached, no new segments will be requested until the buffer arrives at the lower threshold again. To achieve an hysteresis effect a third threshold, somewhere between the lower and upper bound, can also be introduced. Through this, the maximum buffer size can also be controlled. This is important in situations with hard limits on available memory. Mobile devices come to mind here.


\begin{figure}[htb]
	\centering
	\includegraphics[width=0.9\textwidth]{images/R-bufferlevel-flash.pdf}
	\caption{Sample buffer fill level for a \SI{5}{\second} buffered video duration threshold strategy with an additional \SI{2}{\second} initial threshold; \SI{34}{\second} total stalling.}
\label{c3:fig:bufferlevel-flash}
\end{figure}

An example buffer diagram is displayed in Figure~\ref{c3:fig:bufferlevel-flash}. In this case, the initial delay was controlled by a buffered video duration threshold of \SI{2}{\second} and a resume condition also based on buffered video duration but with a \SI{5}{\second} threshold. The strategy produces noticeable less stalls than the null strategy but slightly increases the total stalling time.


%%
\subsubsection{Pacing Strategies}

For segment-based \gls{HTTP} streaming, a two-threshold strategy is not the only supplemental option on top of simple streaming. Here, the controller can pace the request of future segments to match the overall, or the current video bitrate. A safety margin can also be factored in to even out short time fluctuations of either the transmission or the video bitrate. For example, the controller would request segments with an overall transmission rate of 1.25 times the video bitrate. The pacing rate can either be statically chosen in advance or can be calculated dynamically based on current or future conditions. The latter leads to predictive strategies.


%%
\subsubsection{Predictive Strategies}

In predictive strategies, knowledge of the future of the streaming process is used by the controller to adjust the start/stop and segment retrieval conditions. An exact implementation of this strategy would require precise information --- so-called global knowledge --- on future events and therefore it can only be conducted in a theoretical offline manner. Instead of global knowledge, heuristics can instead attempt to approximate an expected future state.

A very simple predictive approach is to prolong the initial delay to the point, that no intermediate buffer underrun and thus no stall will occur. With global knowledge, the controller can start the stream at the earliest possible point in time, thus minimizing the total stalling time while still having only the initial delay.

\begin{figure}[htb]
	\centering
	\includegraphics[width=0.9\textwidth]{images/R-bufferlevel-startdelay.pdf}
	\caption{Sample Buffer fill level for the delayed playback predictive strategy, \SI{33}{\second} total stalling.}
\label{c3:fig:bufferlevel-startdelay}
\end{figure}

Figure~\ref{c3:fig:bufferlevel-startdelay} depicts the time series of a sample implementation of this delayed playback predictive strategy with all necessary stalling occurring upfront.


%%
\subsubsection{Adaptive Strategies}

Most strategies for adaptive streaming are an extension of both the threshold as well as the pacing strategy. However, instead of a simple transmit-or-no-transmit ruleset, they can make much more fine-grained adjustments. The quality of the stream segment to be requested will be chosen depending on the current fill level and drain rate of the buffer. This makes a trade off between maintaining a certain quality level and putting up with increased waiting times, and dropping the quality to a level sustainable at the current transmission rate.

A practical attempt at implementing an adaptive strategy is made in Section~\ref{c6:sec:mobilestreamingtestbed} for the mobile streaming testbed.


\subsubsection{Real World Implementation Examples}

Actual streaming player implementations often do not implement only one of these strategies, but rather combine ideas from several. Herein, thresholds are often set arbitrarily through best practices which are not empirically evaluated. Also, often a trade-off between user perceived quality and the resulting server load is made as a business decision that can further impact the streaming quality. In general, every streaming service practically implements its own playback strategies. This section describes three example applications.

%%
\paragraph{2011 YouTube Flash Player Buffering Strategy}

Google's video streaming site YouTube is constantly changing its appearance and technical makeup. In recent years, YouTube streams are delivered by one of three players: The Website's Flash player, a browser-integrated \acrshort{HTML}5-based player, or custom player implementations in mobile phones, set-top boxes and similar devices. Here, the Flash-based variant used in 2011 is described.

This player used a single threshold strategy with different threshold values for the initial delay and any subsequent stalls. The values were already described in Figure~\ref{c3:fig:bufferlevel-flash}. This model assumes sufficient network conditions in the beginning, requiring only a short initial playback delay to pre-fill the playback buffer. If, however, stalling occurs, then it will buffer longer to keep the stalling frequency down.

\begin{table}[htb]
\caption{Transmission Related Parameters from YouTube's Video URL Setup}
\label{c3:tbl:yturl}
	\centering
	\begin{tabu}{X[l]X}
		\toprule
		\textbf{\gls{URL} Part} & \textbf{Description} \\ 
		\midrule
		\texttt{v$\alpha$.lscache$\beta$.c.youtube.com} &  Cache server involved in the delivery.\\
		\texttt{algorithm=throttle-factor} and \texttt{burst=40} and \texttt{factor=1.25} & Indicates initial burst plus block sending configuration. \\
		\texttt{ratebypass=yes} & Parameter to indicate no rate limiting.\\
		\bottomrule
	\end{tabu}
\end{table}

Furthermore, YouTube employs a proprietary server-side pacing mechanism outside of the control of the streaming player. This is hinted at in the encoding of the \glspl{URL} of the video files and enforced by the video file cache server. Some of those (not user-changeable) parameters are described in Table~\ref{c3:tbl:yturl}. The pacing was in effect for all videos below high definition resolution, but has since then been extended to include all video files. 

\begin{figure}[htbp]
	\centering
	\begin{subfigure}[b]{0.9\textwidth}
		\centering
		\includegraphics[width=\textwidth]{images/R-blocktransfer.pdf}
		\caption{Overall graph.}
		\label{c3:fig:blocktransfer-overall}
	\end{subfigure}

	\begin{subfigure}[b]{0.9\textwidth}
		\centering
		\includegraphics[width=\textwidth]{images/R-blocktransferdetail.pdf}
		\caption{Detail plot of the rate-limited block-sending phase.}
		\label{c3:fig:blocktransfer-detail}
	\end{subfigure}
\caption{Comparison of downloaded and consumed data volume revealing the pacing mechanism used by YouTube.}
\label{c3:fig:blocktransfer}
\end{figure}


The throttling method, which was also observed in \cite{alcock2011afcyt}, limits the transmission to a rate slightly above the average media bitrate. Measurements and the \gls{URL} scheme indicates this to be $1.25$ times the video bitrate. The rate limit is not constant, instead an ON-OFF block-sending scheme is facilitated. The scheme transmits short packet bursts, typically \SI{64}{\kibi\byte} in size, followed by long pauses as seen in Figure~\ref{c3:fig:blocktransfer}. The pause length between two bursts is set dynamically to reach the targeted bitrate on a larger time scale. The initial phase of the stream transmission is conducted unthrottled at line speed, presumably to allow for some pre-buffering to occur at the client's media player. A possible reason for this server-side pacing is to avoid load spikes, with the added side effect of keeping the clients' buffer sizes in check.

Since October 2013 YouTube is serving its \acrshort{HTML}5 videos using \gls{DASH}, albeit without the automatic quality level switches. Additionally, in mid 2014, this \acrshort{HTML} player has now become the default player for all capable browsers, effectively making the Flash player obsolete. Some initial investigations into YouTube's \gls{DASH} traffic were conducted in \cite{2014arXiv1408.5777A}.


%%
\paragraph{Firefox's HTML5 Player Strategy}

Video streaming can also be directly conducted with the Web browser, through the use of a \acrshort{HTML}5 canvas element. The \gls{W3C} specifies the default technical process of \acrshort{HTML}5 video streaming\footnote{\url{http://www.w3.org/TR/html5/embedded-content-0.html\#media-elements}} and essentially suggests a predictive strategy. Herein, the Web browser should estimate and correlate the transmission rate to the video bitrate. A property called ``autoplay'' uses this definition to start playback of the associated video \textit{\enquote{as soon as it can do so without stopping}}. The \acrshort{HTML}5 strategy also allows to limit the buffer size through transmission pacing, negotiated with the server, and appropriately timed range requests.

The open-source Firefox browser represents an implementation of this specification and substantiates it further. The description of this strategy is based on Firefox's version 4.0 released in March 2011. Because it is an online algorithm which does not have global knowledge of the video and transmission speeds of any point in the future it has to estimate these.

\begin{algorithm}[htb]
	\centering
	\begin{algorithmic}
	\IF {$s_{MA} > v_{MA}$} 
		\STATE $c \gets ( b_b=20s \lor b_T=20s )$
	\ELSE
		\STATE $c \gets ( b_b=30s \lor b_T=30s )$
	\ENDIF
	\end{algorithmic}
	\caption{Firefox playback (re-)start decision algorithm.}
\label{c3:alg:firefox}
\end{algorithm}

To estimate the current and future rates, the moving average of the transmission rate $s_{MA}$ and the video bitrate $v_{MA}$ are calculated. The condition $c$ Firefox uses to start and resume the playback process is given in Algorithm~\ref{c3:alg:firefox}, with the buffered video duration $b_b$, and the duration spent buffering $b_T$.

 \begin{figure}[htb]
	\centering
	\includegraphics[width=0.9\textwidth]{images/R-bufferlevel-firefox.pdf}
	\caption{Sample buffer fill level for the Firefox 4 strategy, \SI{44}{\second} total stalling.}
\label{c3:fig:bufferlevel-firefox}
\end{figure}

This approach is quite conservative and trades off long stalling periods for fewer stalls. The test case for the model is shown in Figure \ref{c3:fig:bufferlevel-firefox}. The playback starts only after a long waiting period and intermittent stalls cause a long buffering period. Due to the longer overall stalling time the player needs to buffer more data than other strategies. This may make it unsuitable for devices with sparse amounts of memory, e.g., mobile phones. A large buffer could, on the other hand, also increase the chance of continuous playback in such scenarios with insufficient \gls{QoS}.


\paragraph{Adaptive Streaming Strategies}

Implementations for adaptive streaming players are again mostly proprietary and their behavior has to be derived from measurements. This is even true for protocols with open specifications. These typically only define the transport and media format but explicitly do not specify the players' behavior. \Gls{DASH} is one of these cases.

Microsoft's Silverlight player's strategy is described in \cite{BLTJ:BLTJ20505}. It employs a two-threshold model and rate estimations. When on of the thresholds is reached, the quality will be adjusted by one step upwards or downwards as long as the transmission rate is sufficient.

The buffering behavior of further protocol variants, including Adobe's \gls{HTTP} Dynamic Streaming, Apple's \gls{HTTP} Live Streaming and a sample implementation of \gls{DASH}, are investigated in \cite{Muller:2012:EDA:2151677.2151686,akhshabi2011experimental}.




% \begin{itemize}
% \item The player has been buffering data for at least 30 seconds.
% \item The player has already buffered an amount of data corresponding to 30 seconds of video.
% \item The video download has been completed.
% \item The moving average of the transmission rate is larger than the moving average of the video bitrate and the player has a safety buffer with 20 seconds of video data.
% \end{itemize}

% \begin{table}[htb]
%     \caption{Variables involved in buffering decisions.}
%     \label{c3:tbl:buffvars}
%     \centering
%     \begin{tabu}{|l|X[p]|} 
%     \hline
%     Variable & Description \\ \hline
%     $s_{MA}$ & Moving average of the transmission speed. \\
%     $v_{MA}$ & Moving average of the video bitrate. \\ 
%     $c$   & Condition upon which to start/resume playback. \\
%     $b_b$    & Amount of video data the buffer contains. \\
%     $b_T$    & Amount of time spent in non-playing buffering state. \\ \hline
%     \end{tabu}
% \end{table}

% used yt-delay/hPUGNCIozp0_delay_2500 1, spyder, color #aa5500
% data
% start delay 33s 
% flash 33.82s
% stalling 32.68s
% html5 as implemented in firefox 44s


%Which offers better quality to users? Some approaches (TODO: refs and explain how)

% Longer waiting time but very few stops. Stalling model definitely the shortest waiting time but stops too frequent with insufficient network conditions, i.e. can not really be used. Delayed playback requires total knowledge not available beforehand. Can maybe used with reduced information, bandwidth estimation, but this is essentially HTML5/Firefox.




%``application comfort'' time to skip, ...
%\subsubsection{Unreliable Streaming Metrics}
%... and why they mostly do not work / are not applicable.