%!TEX root = ../../dissertation.tex
%%%%%%%%%%%%%%%%%%%%%%%%%%%%%%%%%%%%%%%%%%%%%%%%%%%%%%%%%%%%%%%%%%%%%%%%%%%%%%%%
\section{Mobile Core Network Load}
\label{c4:sec:loaddefinition}

Now that both the basic architecture and protocols are and introduced and related work is presented, it is time to discuss the specific perspective on the \gls{CN} control plane.

Existing core network measurement studies looked at the control plane mostly in a rather incoherent manner. Some aspects were singled out and presented without forming an overarching motif. The driving question for this chapter was that of core network load. This section attempts to explain the understanding of load in this context. Following afterwards is a discussion on potential factors that could influence this load.


%%
\subsection{Load Definition}

A traditional definition of link load $\rho_{l}$ is the ratio of the used versus the available bandwidth on a link

\begin{equation}
\rho_{l} = \frac{b_{u}}{b_{a}}\text{.}
\end{equation}

The network load $\rho_{N}$ can then simply be defined as the average load of all involved links

\begin{equation}
\rho_{N} = \frac{\sum_{i} \rho_{l,i}}{i}\text{.}
\end{equation}

The link itself is however not the only component, that has a limited capacity and thus can experience load. Actually, a link's capacity is determined by the transmission speed of the interfaces of the two involved network nodes. Any involved node has also other limiting factors, related to the available memory, processing power, or some other resource with fixed capacity.

In Internet core routers those other factors are mostly well known and researched. Their main functionality is to forward packets on the basis of a routing table. This table is generated by exterior and interior gateway protocols, usually \gls{BGP} and \gls{RIP} or \gls{OSPF}, and may grow rather large. The generation and maintenance of a routing table including all lookups --- which might be expensive in a large table --- incurs load on the router's available memory and processing capacity. This kind of load can be attributed to the router's control plane and occurs in addition to the userplane packet forwarding load. All in all, the Internet's control plane is rather lightweight and isolated. Minimal state is kept only where necessary.

The situation is a bit different in a mobile network. Here, the control and user plane are tightly coupled as discussed in Section~\ref{c4:sec:3gpparchitecture}.  Therefore, load of individual resources cannot be looked at separately anymore and a node will be limited by any one of these resources. The load of any particular mobile core network node $\rho_{n}$ is then defined as

\begin{equation}
\rho_{n} = \max(\rho_{l}, \rho_{m}, \rho_{p})
\end{equation}

with the memory load $\rho_{m}$ and processing load $\rho_{p}$. And the total load of the \gls{CN} $\rho_{CN}$ as

\begin{equation}
\rho_{CN} = \max_{i}(\rho_{n,i})\text{.}
\end{equation}

In this definition the performance of the \gls{CN} can limited by any one of the involved core nodes. Looking back at the the \gls{3G} architecture this should be an appropriate definition. With this model definition at hand, one can now attempt to apply it to an actual mobile network and determine the individual load values. But, one additional limitation has to be introduced first reflecting the situation in real mobile networks. Core network nodes should be considered as black boxes. They are custom pieces of hardware sold by vendors as-is, providing no opportunity to directly monitor the inner workings of the node, including memory and CPU usage. Only the network traffic can be directly tapped and investigated as will be described in Section~\ref{c4:sec:methodology} and further. But the amount of signaling traffic exchanged on these observable links could give ample opportunities to indirectly infer some of the nodes' inner state and load.

With the basics of the architecture in mind, a top candidate for being a load bottleneck is the \gls{GGSN}. All traffic leaving or entering the packet switched domain must go through this element, and it is involved in all \gls{CN} \gls{gtp} signaling procedures as well. Being an endpoint for the \gls{gtp} tunnel makes it responsible to sort and encapsulate incoming traffic into the corresponding user tunnel. To accomplish this a lot of state has to be kept and processed when signaling occurs. Therefore, the \gls{GGSN} will be the node under scrutiny in the trace evaluation and performance model creation.

While looking at the \gls{GGSN} may be the most obvious choice, it is by far not the only one. In addition to \gls{gtp} tunnels the \gls{SGSN} acts interface to the radio network as well, which involves handling \glspl{RAB} and mobility management. However, it can be assumed, that the number of \glspl{SGSN} employed in a mobile network is larger than that of \glspl{GGSN}, as they are typically kept closer to the regionally distributed radio networks. This means that a single node would have to handle less mobile devices and related signaling interactions. One has also to bear in mind that the \gls{SGSN} can be completely circumvented by setting up a direct tunnel between \gls{GGSN} and \gls{RNC}.

Apart from the two gateways directly inside the traffic path there are several other nodes essential to the control plane decision making, which may very well be also very load-sensitive. The \gls{HLR} for example is a central database storing all user related information which need to be retrieved any time a user needs to undergo initial authentication and authorization. Typically, the procedures the elements are involved in are fewer and they are also harder to investigate with the data available to us. Hence, it was decided to concentrate just on the case of the \gls{GGSN}.


%%
\subsection{Load Influencing Factors}

With the described understanding of core network load at hand, one can now speculate on factors that could influence the load in mobile networks. Such factors will also play an important role making them targets for the following evaluation.

The first and arguably one of the most important factors are the mobile devices themselves. They are the source of any user traffic and the cause for most signaling procedures, for some procedures directly but for most indirectly. But no device, and the person using it, is identical to another. Therefore, it stands to reason, that the device is only an aggregate of several influence subfactors.

Specifically, this includes the type of device --- defined by the hardware, the baseband as well as the \gls{os} --- and the running applications. They decide when the device should establish a mobile data connection, how long the connection is held, or which mobile technology takes preference. Depending on the \gls{RAT} in use, subtle behavioral and signaling differences can be expected, e.g., in the timing of the radio transmission intervals, which could influence the investigation. Some specific tunnel duration properties could stem from the \gls{os}'s \gls{IP} and transport protocol implementation. For example, \gls{TCP} timeouts might be configured to different default values influencing the duration of transport layer connections and therefore also the underlying tunnels.

The actual user-traffic patterns are of course generated by the applications at the device. The application traffic spectrum ranges from low volume but extremely long duration messaging apps over recurring ad-retrieval up to short duration burst downloads. Since the mobile application ecosystem is so rich and ever growing every device will pose a unique combination of applications. The governing factor in everything device-related is the user and her behavioral patterns. This expresses itself both in the traffic dynamics and in the mobility pattern. But it is neigh impossible to single out individual behavior in a network's traffic mix or a large network trace.

Easier to observe are the temporal statistics of large user groups, not targeting individual users but the overall effects of a device's usage in a certain time span, e.g. based on the time of day or the day of the week. In network user traffic analyses diurnal effects are typically very distinct, with peak traffic some time during the day and the lowest traffic shortly after midnight. But studies investigating this typically only look at user traffic. It should prove interesting to find out if the \gls{CN} control plane shows similar patterns and can be correlated to user traffic.

The second large influence factor are the control plane state machines and related signaling procedures. If a network-side state machine inactivity timer decides to remove an existing tunnel, signaling will occur, which could mean there will be a large number of tunnels with a duration in this range. While most \gls{3G} control plane timers have default values, they are often changed by the manufacturer or network operator and will vary from one implementation to another. It is therefore quite difficult to give any hard numbers in advance, one has to correlate such aspects with certain events in the results.



%%%%%%%%%%%%%%%%%%%%%%%%%%%%%%%%%%%%%%%%%%%%%%%%%%%%%%%%%%%%%%%%%%%%%%%%%%%%%%%%
% \subsection{Factors Influencing Tunnel Durations}

% With such a dataset available and with the intent to evaluate core network signaling by looking at tunnel durations, let's first discuss some of the factors that influence this duration.

% One factor are the mobile devices themselves. The device decides when it should establish a mobile data connection, how long the connection is held, or which mobile technology takes preference. Devices can be further differentiated by their operating system and their firmware (sometimes called \textit{baseband}) which usually takes care of much of layers 1 and 2.

% Some specific tunnel durations could stem from the TCP/IP stack implementations in the operating systems of the devices. TCP timeouts might be configured to different default values in different releases of OSs. Also, mobile network firewalls have been found to interfere with transport and application-layer timeout and keep-alive or heartbeat mechanisms on mobile devices \cite{sigcomm11middleboxes}.

% Of course, the applications that run on top of the OS and generate the actual user-traffic patterns play a role as well. An example for how applications can influence network signaling is the casual game ``Angry Birds'' mentioned before. Since the application ecosystem for smartphones is extremely rich (and grows still), we cannot pinpoint individual ones from our aggregate dataset.

% An additional factor in the picture is the user and his or her behavioral patterns. They express themselves both in the traffic dynamics and in the mobility pattern, but they are rather difficult to distinguish in such a dataset given the large amount of data and the difficulty of correctly correlating tunnel management messages. We leave this as a potential future work.

% We also expect the mobile network and its protocol implementations to express themselves in the measurements. For example, the \gls{RRC} idle timer is typically in the range of 10 to 30 minutes, which could mean there will be a large number of tunnels with a duration in this range. Such choices are usually made either by the mobile network operator or the device manufacturer and can vary from one implementation to another. It is therefore quite difficult to give any hard numbers in advance, and one has to correlate such aspects with certain events in the results.

% Based on these factors, it was decided to make a first categorization according to the device type, be it either a smartphone, a regular or feature phone, or one of the many 3G dongles or mobile routers. Second, we also differentiate based on the device operating system, if known. Both differentiating aspects should prove valuable for example in deciding if currently some phone types put more signaling load on the network and to direct measures to improve this situation. Pitfalls in this differentiation are described in the next sections.



%work in 23.843 \cite{3gpp.23.843} Study on Core Network (CN) overload solutions
%GTP-C retransmission of unacknowledged requests"
%currently: semi-static DNS based load balancing (does this apply only to LTE/SGW?)