%!TEX root = ../../dissertation.tex
%%%%%%%%%%%%%%%%%%%%%%%%%%%%%%%%%%%%%%%%%%%%%%%%%%%%%%%%%%%%%%%%%%%%%%%%%%%%%%%%
\section{Mobile Core Network Load}
\label{c4:loaddefinition}

Now that both the basic architecture and protocols are and introduced and related work is presented, it is time to discuss our specific perspective on the \gls{CN} control plane.

Existing core network measurement studies looked at the control plane mostly in a rather incoherent manner. Some aspects were singled out and presented without forming an overarching motif. The driving question for this chapter was that of core network load. This section attempts to explain the understanding of load in this context. Following afterwards is a discussion on potential factors that could influence this load.


\subsection{Load Definition}

A traditional definition of link load $\rho_{l}$ is the ratio of the used versus the available bandwidth on a link

\begin{equation}
\rho_{l} = \frac{b_{u}}{b_{a}}\text{.}
\end{equation}

The network load $\rho_{l}$ can then simply be defined as the average load of all involved links

\begin{equation}
\rho_{n} = \frac{\sum_{i} \rho_{l,i}}{i}\text{.}
\end{equation}

The link itself is however not the only component, that has a limited capacity and thus can experience load.


% <--

This can be subsumed under the term network ``load'' which we plan to investigate in this work. Therefore, in scenarios such as the ones mentioned above, radio access is not the bottleneck to connectivity any more, but signaling is.

Before beginning the evaluation, the primary question driving this investigation was: ``How can load in a core network be defined and measured?'' A summary of our thoughts to this question follows here.

With the basics of the architecture in mind, a top candidate for high load is the \gls{GGSN}. All traffic leaving or entering the packet switched domain must go through this element, and it is in control of the described GTP signaling procedures as well. Being an endpoint for the GTP tunnel makes it responsible to sort and encapsulate incoming traffic into the corresponding user tunnel. To accomplish this a lot of state has to be kept -- and processed when signaling occurs. Therefore, our working hypothesis is, that in order to determine load the \gls{GGSN} needs to be monitored closely and any traffic related to this node investigated for indications of the current load.

For our definition of the term ``load'' we differentiate between signaling load and overhead on the one hand and processing load and memory consumption on the other hand. Both are measures of load at specific nodes. While the former mostly has an impact on the actual network traffic, the latter can only be grasped inside the network element. With our data we can directly investigate the signaling traffic but indirect measures for the processing load and memory usage have to be found. In the rest of this section we evaluate the results of several approaches to both of these definitions of load.

While looking at the \gls{GGSN} may be the most obvious choice, it is by far not the only one. 
In addition to GTP tunnels the \gls{SGSN} has to handle \gls{RAB} and mobility management as well. However, it is assumed, that there are more regionally distributed \gls{SGSN} nodes present in a typical mobile network. This means that a single element would have to handle less mobile devices and therefore load. One has also to bear in mind that the \gls{SGSN} can be completely circumvented by setting up a direct tunnel between \gls{GGSN} and \gls{RNC}.

Apart from the two gateways directly inside the traffic path there are several other nodes essential to the control plane decision making, which may very well be also very load-sensitive. The \gls{HLR} for example is a central database storing all user related information which need to be retrieved any time a user needs to undergo initial authentication and authorization. Typically, the procedures the elements are involved in are fewer and they are also harder to investigate with the data available to us. Hence, it was decided to concentrate just on the case of the \gls{GGSN}.


%%%%%%%%%%%%%%%%%%%%%%%%%%%%%%%%%%%%%%%%%%%%%%%%%%%%%%%%%%%%%%%%%%%%%%%%%%%%%%%
\subsection{Load Influencing Factors}

Having described our understanding of core network load we can now move to discuss some of the factors that could influence the load, making them targets for our evaluation.

The first and arguably one of the most important factors are the mobile devices themselves. Specifically, this covers the behavior of the network layer 1 and 2 implementation (sometimes called ``'baseband'') as well as the \gls{os} and the running applications. The OS and baseband decide when the device should establish a mobile data connection, how long the connection is held, or which mobile technology takes preference. Depending on the access technology, be it \acrshort{GPRS}, \acrshort{EDGE}, \acrshort{UMTS}, \acrshort{HSPA}, or \acrshort{HSPA+}, we can expect subtle differences through their specifications, e.g. in the timing of the radio transmission intervals, which could influence our investigation. 

Some specific tunnel duration properties could stem from the \gls{os}'s IP and transport protocol implementation. For example, TCP timeouts might be configured to different default values causing mobile connections and tunnels to be held either shorter or longer. Also, mobile network firewalls have been found to interfere with transport and application layer timeout and keep-alive or heartbeat mechanisms on mobile devices \cite{sigcomm11middleboxes}.

The actual user-traffic patterns are generated by the applications running atop the OS. An example for how applications can influence network signaling is the aforementioned ``Angry Birds'' with its ad-retrieval strategy causing network traffic and possibly signaling in certain intervals. Since the application ecosystem for smartphones is extremely rich and ever growing we cannot pinpoint individual ones from our aggregate dataset.

An additional factor in the picture is the user and her or his behavioral patterns. They express themselves both in the traffic dynamics and in the mobility pattern, but they are rather difficult to distinguish in such a dataset given the large amount of data and the difficulty of correctly correlating tunnel management messages. We leave this as potential future work.

Easier to observe are the temporal effects of user behavior, which do not target individual users but the overall effects of a device's usage based on the time of day, the day of the week, or other time spans. In network user traffic analyses diurnal effects are typically very distinct with peak traffic some time during the day and the lowest traffic shortly after midnight. But these investigations are for user traffic only. We aim to find out, if the mobile network control plane shows similar patterns and can thusly be correlated to user traffic.

We also expect the mobile network and its protocol implementations to express themselves in the measurements. For example, the \gls{RRC} idle timer is typically in the range of 10 to 30 minutes, which could mean there will be a large number of tunnels with a duration in this range. Such choices are usually made either by the mobile network operator or the device manufacturer and can vary from one implementation to another. It is therefore quite difficult to give any hard numbers in advance, and one has to correlate such aspects with certain events in the results.


work in 23.843 \cite{3gpp.23.843} Study on Core Network (CN) overload solutions
GTP-C retransmission of unacknowledged requests"
currently: semi-static DNS based load balancing (does this apply only to LTE/SGW?)