%!TEX root = ../../dissertation.tex
%%%%%%%%%%%%%%%%%%%%%%%%%%%%%%%%%%%%%%%%%%%%%%%%%%%%%%%%%%%%%%%%%%%%%%%%%%%%%%%%
\section{Related Work}
\label{c4:relwork}


This chapter is a compilation and extension of previous investigations conducted in \cite{metzger2012research}, \cite{metzger2014jcnc}, and \cite{metzger2014lossmodel}. 

The amount of research conducted in the area of the mobile network control plane is scarce to say the least. No direct predecessor to this work is known. Still, some related work exists, especially if the focus is widened.

In the following sections we divide the related work into four distinct fields.

Work in the first and second sections evaluate properties of the mobile network and its traffic. They are distinguished in their approach to the investigation, as the first group uses active measurements from mobile devices or conclude from other sources of traffic whereas to the other one has access to passive measurements from inside a \gls{3G} mobile network. Publications from the third category can be generally subsumed under the term ``traffic modeling'' and may not be specific to cellular networks. The final field concerns itself with the overall investigation of mobile network commonalities not falling into one of the previous specific categories.

The investigations conducted here do not strictly fall into either one of these but instead aims to provide diverse insights into the control plane from the perspective of the core network. We present a selection of publications from these fields and detail the interesting aspects for this work.


%%
\subsection{Device Active Measurement Investigations}

The approach taken by active measurement studies is simple yet still very insightful. They are performed by writing custom application layer measurement programs for a mobile device. Specific traffic patterns are then generated, recorded, and evaluated. While this can provide very detailed information about the higher network layers, it is limited both in lower layer information as well as scale, due to being limited to a rather low number of devices.

Despite being more ore less completely specified in the \gls{3GPP} documents, there is no open layer 1 and 2 (together also called ``baseband'') implementation for \gls{3G}.\footnote{Apart from OsmocomBB (\url{http://bb.osmocom.org/trac/}), but it only provides \gls{GSM} and partial \gls{GPRS} functionality.} Therefore, the baseband's behavior can not be directly measured from the application layer, but attempts to infer some properties are still worth making as the following selection of publication demonstrates.

Xu et al. use data from a location service combined with active measurements to determine the possible geographic location of a \gls{GGSN} in order to improve the location of application content caches for the current network infrastructure. \cite{Xu:2011:CDN:2007116.2007149}. Similarly, Wang et al. in \cite{sigcomm11middleboxes} developed a program to probe mobile networks for middle boxes. That term includes any node, that alters traffic and affects performance not intended by the actual end-to-end protocols. Examples are \gls{CGN} \cite{rfc7021}, firewalls, or intercepting \gls{HTTP} proxies. A large number of such nodes were present in the investigated mobile networks and resulted in increased device power usage and download durations and even pose security issues themselves.

Concerning methods to infer specific baseband and \gls{RRC} state machine timer values with active measurements, a 2007 paper~\cite{4640935} presents a way to do this by transmitting packets with a varying inter-departure time and studying the resulting arrival pattern. Indeed, the dynamics of the radio interface's \gls{RRC} signaling and involved state machines are under investigation by several publications. However, almost all focus solely on the impact at the radio interface but pay little attention to potential implications in the \gls{CN}.

The aforementioned work is continued in \cite{5360763} and uses the presented tools to derive \gls{RRC} transitions and power usage from traffic patterns. They found, that operators have a rather larger freedom in configuring the state machines, deviating from the standard and even omitting some states completely.

A further example of cross-layer influences in mobile cellular networks is \cite{qian2011profiling}. It discusses the impact of application layer behavior on \gls{RRC} signaling and its consequences for device energy consumption and radio channel allocation efficiency. The authors argue that there is much room for improvement in this area, and propose some enhancements.

This is further elaborated on by research from Schwartz et al.\cite{schwartz2013angrybirds} using the same technique to analyze the radio signaling load and thus power efficiency from several mobile phone applications. The impact of custom set state machine timers interacting with application traffic is further investigated and the \gls{QoE} is investigated.


%%
\subsection{Research Based On Network Traces}

The second alternative to mobile network investigations comes in the form of recording and evaluation traffic traces inside the network. This brings a much larger experiment scale with it, albeit usually at the cost of some finer grained details in the higher protocol layers because of aggregation to flow level. 
With core network measurements, the signaling traffic of the observed link can also be directly investigated, which is a huge benefit compared to the guesswork in active measurements.

The authors of \cite{4675847} investigate the influence of individual \gls{CN} nodes on the one-way delay distribution of user traffic packets. According to the work, the latency portion added by the \gls{SGSN} is larger but also fluctuating more, while the \gls{GGSN} added a small but steady amount of latency. This provides us with initial clues on the expected load impact of the \gls{CN} for our own investigation.

Following up on the topic of mobile network one-way delays is Laner et al. in \cite{laner2012delaycomparison}. The end-to-end latency of a very early \gls{LTE}/\gls{EPC} network implementation is compared to that of a \gls{HSPA} network at several measurement points in the networks. The results show a lower median latency for \gls{LTE}, despite some scenarios still being in favor of \gls{3G} networks.

The authors of \cite{Shafiq:2012:FLC:2254756.2254767} limit their focus to a specific subset of connected devices, namely those of \gls{M2M} type. These are small automated devices, that periodically send out data, e.g. sensor readings, or receive control commands. The paper attempts to characterize these on the basis of their generated mobile network traffic. The patterns are clearly distinguishable from traffic caused by other device types such as smartphones.

A 2012 publication~\cite{Zhang:2012:UCC:2377677.2377764} presents us with a more general look on the traffic composition of cellular access networks in comparison to wired access network. Much more and shorter flows are occurring in the case of cellular networks.
It will be interesting to see if this shorter-but-more theme is also evident in signaling traffic. Additionally, even traffic pattern distinctions between types of applications are made showing a wide range of possible outcomes across the investigated applications.

Both The authors of \cite{shafiq2011characterizing} and \cite{paul2011understanding} take the approach of looking at high-level user traffic characteristics in a mobile network, focusing on temporal and spatial variations of user traffic volume and peeking at the influence of different devices on this metric. 


Two parallel approaches to network data analysis ; \cite{baer2011two} delivers a theoretical introduction on how to conduct large scale network measurements and compares some data evaluation approaches.

%% <-

Traffic analysis at short time-scales: an empirical case study from a 3G cellular network \cite{4570772} METAWIN based but not investigating signaling


\gls{RRC} state machine:
uses simulations based on wifi and synthetic traces
Based on the state both procedures can enable and disable radio tunnels as well as core network tunnels, making them a good example of user traffic dynamics directly influencing core network signaling, similar to the observations in \cite{lee2007detection}. 
We identify user traffic dynamics as one vector to influence core network signaling, similar to the observations in \cite{lee2007detection}.
In \cite{lee2007detection}, mobile network traces are used to simulate a malicious signaling storm by transmitting low-volume user plane traffic with inter-departure times slightly larger than the transition timers in the \gls{RRC} state machines. This constantly causes signaling to occur. The authors propose tools to detect this, and discuss a possible magnitude of this type of \gls{DoS} attack.


In 2006, Svoboda et al. \cite{svoboda2006composition} conducted a core network measurement study of various user traffic related patterns, and also provided an initial insight into \gls{PDP} context activity and durations.
In 2006, a core network measurement study of various user traffic related patterns was conducted \cite{svoboda2006composition}, providing an initial insight into \gls{PDP} context activity and durations.


Another recent publication at \cite{he2012panoramic} provides an investigation aimed at \gls{RRC} signaling on the \gls{RNC} to \gls{SGSN} link but not at \gls{gtp} signaling at the \gls{SGSN} to \gls{GGSN} path which we deem more important for our core network load characteristics research. The authors classify their evaluations based on device model and vendor and on the application type, and find that different devices have strongly different \gls{RRC} characteristics, which could possibly also have an impact on \gls{gtp} signaling. Here the \gls{RRC} evaluation was done in a direct manner using explicit logs from the \gls{RNC}. 
 \cite{he2012panoramic} provides an investigation aimed at radio network signaling. The authors find that different devices have also different signaling characteristics.


A 2010 publication\cite{Qian:2010:CRR:1879141.1879159} however uses the indirect \gls{RRC} inferring method described earlier on a core network TCP trace data set and finds that the involved \gls{RRC} state machine is largely inefficient in terms of signaling overhead and energy consumption for typical traffic patterns seen in the data.
A 2010 publication \cite{Qian:2010:CRR:1879141.1879159} indirectly infers radio signaling from TCP traces concluding that very commonly occurring traffic patterns cause large signaling overhead and high energy consumption.


%%
\subsection{Traffic Modeling}

\begin{itemize}
	\item Source traffic modeling of wireless applications \cite{staehle2000source}
	\item Traffic modeling and characterization for UMTS networks \cite{965876}
\end{itemize}



%%
\subsection{General Mobile Network (Infrastructure) Investigations}
\begin{itemize}
	\item Comparative Performance Study of LTE Downlink Schedulers \cite{biernacki2013ltescheduler}
	
	\item 22.801 \cite{3gpp.22.801} Study on non-MTC mobile data applications impacts; for angry birds // relwork important
	\item 23.843 \cite{3gpp.23.843} Study on Core Network (CN) overload solutions
	\item 24.826 \cite{3gpp.24.826} Study on impacts on signalling between User Equipment (UE) and core network from energy saving; deals mostly with switching off cells and moving over UEs, not actual core network efficiency
	\item 29.807 \cite{3gpp.29.807} Study on GTP-C overload control mechanisms
\end{itemize}

A final paper \cite{Ricciato2010551} presents some \gls{DoS} attack scenarios on these networks from a theoretical view. As a \gls{DoS} either needs to find a weak (performance-wise) link in an architecture or a good source for an amplification attack -- small information causes a large amount of information to be computed or transmitted -- this is also very helpful information in evaluating core network load and finding bottlenecks.

Correlation to stories about carrier complaints over (radio) ``signalling storms''  and 3gpp R8 Fast Dormancy \cite{3gpp.25.331} and \cite{gsma2011fdbestpract}





All of these touch to some degree parts of the areas tackled in this paper, but we think that the combination of the focus on core signaling, a statistical evaluation of PDP Contexts with an investigation of sources influencing these, and a simple load model are genuine contributions of our work.
