%!TEX root = ../../dissertation.tex
%%%%%%%%%%%%%%%%%%%%%%%%%%%%%%%%%%%%%%%%%%%%%%%%%%%%%%%%%%%%%%%%%%%%%%%%%%%%%%%%
\section{Related Work}
\label{c4:sec:relwork}

The investigations conducted in this chapter do not fall strictly into an existing research category but instead aim to provide diverse insights into the control plane from the perspective of the core network. Nonetheless, a selection of publications from the tackled fields is collected here and the interesting aspects for this work are noted. In the following sections the related work is divided into four distinct fields.

Work in the first and second sections evaluate properties of the mobile network and its traffic. They are distinguished in their approach to the investigation, as the first group uses active measurements from mobile devices or conclude from other sources of traffic whereas to the other one has access to passive measurements from inside a \gls{3G} mobile network. Publications from the third category can be generally subsumed under the term \textit{traffic modeling} and may not be specific to cellular networks. The final field concerns itself with investigative work conducted by the responsible standardization and organizational bodies themselves, i.e., the \gls{3GPP} and \gls{GSMA}.


%%
\subsection{Device Active Measurement Investigations}

The approach taken by active measurement studies is simple yet still very insightful. They are performed by writing custom application layer measurement programs for a mobile device. Specific traffic patterns are then generated, recorded, and evaluated. While this can provide very detailed information about the higher network layers, it is limited both in lower layer information as well as scale, due to being limited to a rather low number of devices.

Despite being more or less completely specified in the \gls{3GPP} documents, there is no open layer 1 and 2 (together also called ``baseband'') implementation for \gls{3G}.\footnote{Apart from OsmocomBB (\url{http://bb.osmocom.org/trac/}), but it only provides \gls{GSM} and partial \gls{GPRS} functionality.} Therefore, the baseband's behavior can not be directly instrumented from the application layer. Attempts to infer some properties are still worth conducting as the following selection of publication demonstrates.

In~\cite{Xu:2011:CDN:2007116.2007149} Xu et al.\ use data from a location service combined with active measurements to determine the possible geographic location of a \gls{GGSN} in order to improve the location of application content caches for the current network infrastructure. Similarly, in \cite{sigcomm11middleboxes} Wang et al.\ developed a program to probe mobile networks for middle boxes. That term includes any node, that alters traffic and affects performance not intended by the actual end-to-end protocols. Examples are \gls{CGN}~\cite{rfc7021}, firewalls, or intercepting \gls{HTTP} proxies. A large number of such nodes were present in the investigated mobile networks and resulted in increased device power usage and download durations and even pose security issues themselves.

Concerning methods to infer specific baseband and \gls{RRC} state machine timer values with active measurements, a 2007 paper~\cite{4640935} presents a way to do this by transmitting packets with a varying inter-departure time and studying the resulting arrival pattern. Indeed, the dynamics of the radio interface's \gls{RRC} signaling and involved state machines are under investigation by several publications. However, almost all focus solely on the impact at the radio interface but pay little attention to potential implications in the \gls{CN}.

The aforementioned work is continued in \cite{5360763} and uses the presented tools to derive \gls{RRC} transitions and power usage from traffic patterns. They found, that operators have a rather larger freedom in configuring the mobile network control plane state machines and deviate from the standard and even omit some states completely.

A further example of cross-layer influences in mobile cellular networks is \cite{qian2011profiling}. It discusses the impact of application layer behavior on \gls{RRC} signaling and its consequences for device energy consumption and radio channel allocation efficiency. The authors argue that there is much room for improvement in this area, and propose some enhancements.

This is further elaborated on by research from Schwartz et al.\cite{schwartz2013angrybirds} using the same technique to analyze the radio signaling load and thus power efficiency from several mobile phone applications. The impact of custom set state machine timers interacting with application traffic is further investigated and the \gls{QoE} is investigated.


%%
\subsection{Research Based On Network Traces}

The second alternative to mobile network investigations comes in the form of recording and evaluation traffic traces inside the network. This brings a much larger experiment scale with it, albeit usually at the cost of some finer grained details in the higher protocol layers because of aggregation to flow level. With core network measurements, the signaling traffic of the observed link can also be directly investigated, which is a huge benefit compared to the guesswork in active measurements.

The authors of \cite{4675847} investigate the influence of individual \gls{CN} nodes on the one-way delay distribution of user traffic packets. According to the work, the latency portion added by the \gls{SGSN} is larger but also fluctuating more, while the \gls{GGSN} added a small but steady amount of latency. This provides us with initial clues on the expected load impact of the \gls{CN} for the investigations in this work.

Following up on the topic of mobile network one-way delays is Laner~et~al.\ in \cite{laner2012delaycomparison}. The end-to-end latency of an early \gls{LTE}/\gls{EPC} network implementation is compared to that of a \gls{HSPA} network at several measurement points in the networks. The results show a lower median latency for \gls{LTE}, despite some scenarios still being in favor of \gls{3G} networks.

The authors of \cite{Shafiq:2012:FLC:2254756.2254767} limit their focus to a specific subset of connected devices, namely those of \gls{M2M} type. These are small automated devices, that periodically send out data, e.g., sensor readings, or receive control commands. The paper attempts to characterize these on the basis of their generated mobile network traffic. The patterns are clearly distinguishable from traffic caused by other device types such as smartphones.

A 2012 publication~\cite{Zhang:2012:UCC:2377677.2377764} presents us with a more general look on the traffic composition of cellular access networks in comparison to wired access network. Much more and shorter flows are occurring in the case of cellular networks. It will be interesting to see if this shorter-but-more theme is also evident in signaling traffic. Additionally, even traffic pattern distinctions between types of applications are made showing a wide range of possible outcomes across the investigated applications.

Both the authors of \cite{shafiq2011characterizing} and \cite{paul2011understanding} take the approach of looking at high-level user traffic characteristics in a mobile network, focusing on temporal and spatial variations of user traffic volume and peeking at the influence of different devices on this metric. Additionally, \cite{baer2011two} delivers a theoretical introduction on how to conduct large scale network measurements and compares some data evaluation approaches. The 2008 paper of \cite{4570772} takes a look at times scales and time of day deviations observed in aggregated user traffic in a mobile network.

Up until now no trace-based investigation considered the control plane in their evaluation. The following publications include this at least to some degree.

In 2006, Svoboda~et~al.~\cite{svoboda2006composition} conducted a core network measurement study of various user traffic related patterns, and also provided an initial insight into \gls{PDP} context activity and durations. Another paper~\cite{lee2007detection} combines simulations based on WiFi and synthetic traces with prior knowledge of \gls{RRC} states and their effects to investigate detection methods for signaling \gls{DDoS} occurring on the radio interface. A possible magnitude of this type of attack is discussed. This also gives an indication of the correlation between user traffic patterns and radio signaling.

A 2010 publication\cite{Qian:2010:CRR:1879141.1879159} uses the indirect \gls{RRC} inferring method described earlier on a core network \gls{TCP} trace data set and finds that the involved \gls{RRC} state machine is largely inefficient in terms of signaling overhead and the device's energy consumption for the traffic patterns seen in the data. 

A more recent publication at \cite{he2012panoramic} performs a \gls{RRC} investigation at the path between \gls{RNC} and \gls{SGSN}. The authors classify their evaluations based on device model and vendor and on the application type, and find that different devices have strongly different \gls{RRC} characteristics, which could possibly also have an impact on \gls{gtp} signaling. Here, the \gls{RRC} evaluation was done in a direct manner using explicit logs from the \gls{RNC}. A final paper~\cite{Ricciato2010551} recaps some general attack scenarios on \gls{3G} networks that exploit the specific \gls{3GPP} system design. These are often closely related to the control plane.


%%
\subsection{Traffic Modeling}

Extracting viable models from mobile traffic measurements will also play a significant role. The first related work is a survey of source modeling approaches for \gls{GPRS} user traffic from the year 2000 \cite{staehle2000source}. Models for \gls{HTTP} traffic and user behavior are compared and a combined model is recommended. One has to keep in mind, though, that due to the rapid developments in the Web in recent years those models might no longer be valid. 

Similarly, the authors of \cite{965876} derive a synthetic \gls{UMTS} traffic model from wired dial-up traces. By using a batch Markovian arrival process they characterize session traffic in most cases with a lognormal distribution.

Work conducted in \cite{Halepovic:2005:CMU:1089803.1089969} derives a model for the users' mobility in a mobile network. The mobility model is however more focused on the circuit-switched voice communication features of a phone. Likewise, the authors of \cite{Pesch2005385} introduce a traffic model for \gls{SIP} \gls{VoIP} communication in \gls{UMTS} networks. However, this model is specific to the \gls{IMS} domain of \gls{UMTS} and potentially not applicable to the more common over-the-top pure \gls{SIP} traffic. But the model additionally investigates some initial \gls{UMTS} control plane timing values, such as the processing time of \gls{PDP} context activation messages.

A further publication in 2005 \cite{Landman200568} attempts to model the delay of \gls{IP} packets passing through an \gls{UMTS} network using a batch Markovian arrival process. However, the model specifically focuses solely on the delay originating from processing at the radio link and not at the core nodes.

Finally, a further paper by Laner~et~al.~\cite{6214330} investigates, amongst other things, a user's session duration and throughput in a \gls{HSDPA} network. The duration is modeled as an exponential distributions and the throughput using a lognormal distribution, albeit both exhibit additional heavy tail characteristics.


%%
\subsection{\texorpdfstring{\acrshort{3GPP}}{3GPP} and \texorpdfstring{\acrshort{GSMA}}{GSMA} Related Work}

The two associations related to the mobile network under scrutiny, the \gls{3GPP} as well as the \gls{GSMA} themselves have also released some studies and recommendations concerning potential effects of and issues with the control plane. 

In reaction to the mentioned \gls{RRC} signaling \gls{DDoS} the \gls{GSMA} released some best practices \cite{gsma2011fdbestpract} intended to reduce the number of signaling messages in these circumstances. The cause of the \gls{DDoS} were in most cases mobile devices, that circumvented the \gls{RRC} state transition timers and explicitly switched the radio to the idle state after a transmission was finished and re-enables it whenever needed. This can greatly reduce the power usage but increases the number of signaling messages to be sent and thus the load in the radio network and possibly also inside the \gls{CN}. With the presented ``Fast Dormancy'' mechanism, mobile devices are supposed to reduce the radio signaling amount while still saving energy. The implications of this mechanism on the core are not investigated.

A \gls{3GPP}-released study \cite{3gpp.22.801} also describes the diverse traffic mix originating from modern smartphones and its associated signaling problems.

The aim of the study in \gls{TS}~23.843~\cite{3gpp.23.843} is to document some of the control plane bottlenecks and attack vectors on the \gls{CN}. This also includes the interesting case of \gls{GTP-C} overload and causes for this scenario. Some approaches to alleviate the effects are also presented but mostly targeted at the \gls{EPC}. The final study is an extension to the last one \cite{3gpp.29.807} and focuses solely on \gls{GTP-C} overload control to be included in a future version of the \gls{3GPP} architecture. Therefore, the mostly unfinished document again targets just at a future version of \gls{LTE} and provides no investigation of the actual load situation in current \gls{3G} networks.

All of the presented publications relate to some degree to this chapter. However, the combination of the aspects of \gls{CN} signaling with a statistical evaluation and load modeling of \gls{PDP} contexts should be a genuine contribution of this chapter.



%24.826 \cite{3gpp.24.826} Study on impacts on signalling between User Equipment (UE) and core network from energy saving; deals mostly with switching off cells and moving over UEs, not actual core network efficiency