%!TEX root = ../../dissertation.tex
%%%%%%%%%%%%%%%%%%%%%%%%%%%%%%%%%%%%%%%%%%%%%%%%%%%%%%%%%%%%%%%%%%%%%%%%%%%%%%%%
\section{Queuing Theory Basics}

\begin{itemize}
\item Kleinrock Queuing Systems Volume 1 \cite{Kleinrock:1975:TVQ:1096491}
\item Tran-Gia Analytische Leistungsbewertung verteilter Systeme \cite{trangia-lbvs}
\item Markov Models 
\item Solvability and Queuing Simulation
\end{itemize}


\subsection{Little's Law}
``A proof for the queuing formula: L= $\lambda$W'' \cite{little1961proof}

With $L$  as the number of customers in a stable system, the arrival rate of new customers $\lambda$ and the average time $W$ of a customer in the system this universal law states:

\begin{equation}
L = \lambda W
\end{equation}

\subsection{Kendall's Notation}

Kendall's notation is a naming and classification convention for queuing systems first defined by Kendall in in 1953 \cite{kendall1953stochastic} and later extended on. In its simplest form it reads \textit{A/S/s} with A denoting the arrival distribution, S the service time, and s the number of servers. One extended notation \textit{A/S/s-q}, the one we will use, additionally describes the queue length. With this, one can, e.g., easily distinguish between a queueing system ($q=\infty$) and a blocking or loss system ($q=0$). The most commonly used arrival processes and servie time distributions are summarized in Table~\ref{c2:tbl:kendalldistributions}.


\begin{table}[htbp]
	\caption{Typical abbreviation of processes in Kendall's notation.}
	\label{c2:tbl:kendalldistributions}
	\begin{tabu}{|l|X[p]|}
	\hline
	Symbol & Description \\ \hline
	M & Markovian, i.e. Poisson, arrival process or exponential service time distribution\\
	D & Deterministic arrival process or service time distribution\\
	G & General arrival process or service time distribution with no special assumptions\\
	GI & General arrival process with independent arrivals; also called regenerative \\ \hline
	\end{tabu} 
\end{table}

The simplest queuing system is \textit{M/M/1-$\infty$}, which can also be described as a Markov chain and thus 

TODO: oder direkt M/M/n-$\infty$?

State probability, i.e. number of customers in the system
Blocking probability $p_B$ (for loss systems)


Following from Little's Theorem, the queue utilization $\rho$ is given as
\begin{equation}
\rho = \frac{\lambda}{\gamma}
\end{equation}

with $\lambda$ Poisson arrival rate, $\gamma$ exponential service time parameter


explain Erlang loss system and tractability