%!TEX root = ../dissertation.tex
%%%%%%%%%%%%%%%%%%%%%%%%%%%%%%%%%%%%%%%%%%%%%%%%%%%%%%%%%%%%%%%%%%%%%%%%%%%%%%%
\chapter*{Abstract}
\addcontentsline{toc}{chapter}{Abstract}%

The usage of both video streaming as well as mobile networks has become prevalent today. Both make up a large portion of the Internet's current traffic mix. Additionally, video streaming has changed radically changed in recent years, shifting from traditional approaches running atop the unreliable \acrshort{UDP} to using the reliable \acrshort{TCP} as transport protocol. The behavior of these reliable streaming approaches can be much harder to predict because of the larger number of influence factors and cross-correlations between protocol layers. This is especially important in light of the complexity mobile networks are adding to the stack, even more so when the mobile control plane is factored in.

The purpose of this thesis is to evaluate this exact scenario: reliable streaming in mobile networks. The work takes a two-fold approach by first independently investigating both the mobile core network control plane as well as reliable streaming. For reliable streaming, a complete emulation-based measurement framework is introduced and measurement series that highlight the efficiency of certain playback strategies are conducted. Concerning mobile networks, a network trace from a live network is evaluated and explored for characteristics of the core network control plane. Through this, a network control plane load model based on \acrshort{gtp} tunnel statistics is formulated and further evaluated through simulations.

Afterwards, the independent efforts are merged back together again with a study on the influences of network layers on streaming protocols. Many possible sources of influence on the quality of video streaming are discussed and a cross-layer information exchange framework, that can alleviate some of these negative factors, e.g., handover events in mobile networks, is presented. 

Measuring anything related to this topic of reliable streaming in mobile networks is often a difficult task. But the efforts conducted here show that many of the factors related to it require a large amount of understanding and investigation. Therefore, the final part presents two further viable approaches to mobile streaming measurements: a mobile device active measurement environment that facilitates additional sensor and mobility data from the device, and lastly a reliable streaming simulation framework running atop a simulated \acrshort{LTE} network to better evaluate mobility influences on streaming. The latter method is then ultimately used to show the effects of handover events on streaming. 

All in all, this work gives insights into the relationships between reliable streaming and mobile networks. Measurement frameworks and models help to properly investigate them and deal with influencing factors surrounding the issue.


\chapter*{Zusammenfassung}
\begin{german}
Die Nutzung von Videostreaming als auch von Mobilfunknetzen ist in den letzten Jahren stark gestiegen und beide sind nun für einen Großteil des aktuellen Verkehrsmixes im Internet verantwortlich. Jedoch hat sich die Ausprägung des Videostreamings auch sehr stark im Vergleich zu früheren Formen geändert. Wo früher fast ausschließlich das unzuverlässige Transportprotokoll \acrshort{UDP} zur Verwendung kam wird heute mehrheitlich auf das zuverlässige \acrshort{TCP} gesetzt. Das Verhalten von zuverlässigen Streaming kann allerdings durch die größere Menge an Einflussfaktoren und Abhängigkeiten zwischen den Protokollschichten deutlich schwerer zu vorhersagen sein. Dieser Umstand fällt speziell bei Mobilfunknetzen und ihren Komplexen Protokoll- und Signalisierungsstrukturen ins Gewicht, insbesondere wenn zusätzlich auch die ``Control Plane'' der Mobilfunknetze betrachtet wird.

Genau dieses Szenario --- Streaming mit zuverlässigen Transportprotokollen in Mobilfunknetzen --- soll in dieser Dissertation vertieft untersucht werden. Dabei wird ein zweigleisiger Ansatz verfolgt: Zuerst werden Mobilfunknetze und zuverlässiges Streaming getrennt evaluiert. Für die Streaming-Untersuchung wird ein vollständiges emulationsbasiertes Messsystem eingeführt. Mit diesem werden Messreihen, die die Leistung von individuellen Abspielstrategien bewerten, durchgeführt. Mobilfunknetze werden insbesondere bezüglich ihrer Eigenschaften der Control Plane im Kernnetz untersucht. Dies erfolgt aufgrund eines Datensatzes eines Mitschnittes aus einem produktiven Mobilfunknetz und führt zu einer Definition von Last, die sich aus den Eigenschaften von \acrshort{gtp}-Tunneln ableitet. Dieses Lastmodell wird dann durch statische Auswertungen und Simulationen weiter untersucht.

Anschließend verbinden sich diese beiden eigenständigen Forschungsansätze wieder in einer Studie zu den Einflussfaktoren von beispielsweise Netzwerkschichten auf Streaming-Protokolle. Hierbei werden viele Einflussquellen und auch ein möglicher Weg, diese durch einen Cross-Layer Informationsaustausch zu eliminieren oder zumindest zu reduzieren, diskutiert. Beispielsweise können durch diesen Ansatz die negativen Auswirkungen von Handover-Ereignissen auf Video-Streaming abgefangen werden.

Das Messen und Auswerten von zuverlässigen Mobil-Streaming ist in der Regel durch diverse Faktoren ein schwieriges Bestreben. Allerdings zeigen die hier vorgestellten Faktoren auch, dass eine große Menge an Untersuchungen nötig ist, um die Thematik hinreichend zu verstehen. Daher beleuchtet ein abschließender Abschnitt zwei weitere mögliche Messansätze: Einerseits eine aktive Messumgebung, die direkt auf einem mobilen Endgerät läuft und möglichst viele Sensorwerte und Protokollzustandsvariablen des Gerätes mit in eine Messung einbeziehen kann.Dies ist bei Mobilfunkmessungen, die auch häufig einer großen Mobilität und Einflussfaktoren durch den Benutzer unterliegen, ein essentieller Umstand. Zweitens wird eine zuverlässige Video-Streaming-Simulation auf einem vorhandenen Mobilfunksimulator implementiert und dieser genutzt um exemplarisch den Einfluss von horizontalen Handovers auf eine laufende Streaming-Übertragung zu untersuchen.

Zusammenfassend gibt diese Arbeit Einblicke in die Beziehung zwischen zuverlässigen Streaming und Mobilfunknetzen. Die vorgestellten Mess- und Modellierungsmethodiken helfen bei der korrekten Analyse der vielen beobachtbaren Einflussfaktoren und führen letztlich zu einem besseren Verständnis der Thematik.
\end{german}