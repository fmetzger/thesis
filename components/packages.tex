%!TEX root = ../dissertation.tex
%%%%%%%%%%%%%%%%%%%%%%%%%%%%%%%%%%%%%%%%%%%%%%%%%%%%%%%%%%%%%%%%%%%%%%%%%%%%%%%
% packages

\usepackage{ifluatex}
\ifluatex
  \usepackage{fontspec}
  \usepackage{microtype}
  %\usepackage{amsmath}
  \usepackage{amssymb}
  \usepackage{mathtools}
  \usepackage{lualatex-math} % fixes amsmath/mathtools ... for luatex
  \usepackage{unicode-math} % fixes math
  \setmainfont[SmallCapsFont={Linux Libertine O},SmallCapsFeatures={Letters=SmallCaps},Ligatures=TeX]{Linux Libertine O}
  \newfontfamily\scfont[Letters=SmallCaps, Ligatures=TeX]{Linux Libertine O}
  %\usepackage{tex-gyre-math}
  %\setmathfont{texgyrepagellamath-regular.otf} % math font for libertine
  %\setmathfont{texgyrepagella-math.otf} % math font for libertine
  \setmathfont[math-style=ISO,bold-style=ISO,vargreek-shape=TeX,Ligatures=TeX]{TeX Gyre Pagella Math}
  \usepackage{polyglossia} % babel replacement for use with fontspec
  \setdefaultlanguage{english}
\else
  \usepackage[utf8]{inputenc}                   
  \usepackage[T1]{fontenc}
  \usepackage[american]{babel}
  \usepackage{amsmath}
  \usepackage{mathtools} % instead of amsmath; amssymb and amsthm not included
  \usepackage{libertine}
  \usepackage[libertine]{newtxmath} % math font for libertine
  \usepackage{microtype}
  \microtypesetup{expansion=false}%
\fi

\usepackage[hyphens]{url}

\linespread{1.05}
\usepackage[l2tabu,orthodox,abort]{nag} % old style latex & warnings as errors
\usepackage{scrhack} % fix komascript floattolist issues
\usepackage[defernumbers=true, style=alphabetic, citestyle=alphabetic, firstinits=true, backend=biber, doi=true, url=true, block=ragged, maxnames=6]{biblatex} % 
\usepackage{csquotes} % context sensitive quotations; especially for use with biblatex and biber
\usepackage{listings} % ?
\usepackage[scale=3]{ccicons} % creative commons icons
\usepackage{import} % for subimport and relative directory \includegraphics
\usepackage{caption} % figure caption control
\usepackage{subcaption} % sub figure replacement
\usepackage{multirow} % tables spanning multiple rows
\usepackage{longtable} %required for tabu with longtabu
\usepackage{tabu} %supersedes tabularx, provides tables spanning multiple pages
\usepackage{tikz}
\usetikzlibrary{arrows}
\usepackage{marginnote} % ?
\usepackage{enumitem} % ?
\usepackage{array} % ?
\usepackage{styles/tangocolors}
\usepackage[colorlinks, unicode=true]{hyperref}
\hypersetup{pdfencoding=auto, colorlinks=true, linktocpage=true, breaklinks=true, bookmarksnumbered, bookmarksopen=true,bookmarksopenlevel=1, pdfhighlight=/O, urlcolor=webbrown, linkcolor=RoyalBlue, citecolor=webgreen}

\makeatletter
\AtBeginDocument{
  \hypersetup{
    pdftitle = {\@title},
    %pdfauthor = {\@author}
  }
}
\makeatother

%%%%
% glossaries
\usepackage[acronym, style=tree, toc, section=chapter, nogroupskip=true]{glossaries}
% define a custom tree-based style with user1-key support for urls and other references
% TODO indent the par!
\newglossarystyle{treewithref}{%
  \setglossarystyle{tree}% base this style on the list style
  \renewcommand{\glossentry}[2]{%
  \hangindent0pt\relax
  \parindent0pt\relax
  \glsentryitem{##1}\textbf{\glstarget{##1}{\glossentryname{##1}}}%
  \ifglshassymbol{##1}{\space(\glossentrysymbol{##1})}{}%
  \space\glossentrydesc{##1}\glspostdescription%
  \ifglshasfield{user1}{##1}{\\\textit{\glsentryuseri{##1}\glspostdescription}}{}\space##2\par
  }%
}
\setglossarystyle{treewithref}
%\glsdisablehyper   % enable this, if hyperref for glossaries breaks again
%%%%


\usepackage{textcomp} % extra symbols; used for euro and copyright symbol
\usepackage[load-configurations={abbreviations,binary}]{siunitx}
\sisetup{load-configurations = abbreviations,binary-units}
\usepackage{xspace} % ?
\usepackage{algorithmic}
\usepackage{algorithm}

\usepackage[24hr,iso]{datetime}
\usepackage{gitinfo}


%%%%%%%%%%%%%
% classicthesis replacement code
\usepackage[automark]{scrpage2}

% \toprule and other table stuff
\usepackage{booktabs} 

% microtype and pdfspacing
\DeclareRobustCommand{\spacedallcaps}[1]{\textls[160]{\MakeTextUppercase{#1}}}%
%\DeclareRobustCommand{\spacedlowsmallcaps}[1]{\textls[80]{\scshape\MakeTextLowercase{#1}}}%

% typearea
%\usepackage[top=2.5cm, bottom=3.5cm, left=3.5cm, right=2.5cm]{geometry} % redefine page margins
\usepackage[a4paper]{geometry} % redefine page margins
%\usepackage[maxfloats=48]{morefloats} % enable this if too many unprocessed floats are happening

% Disable single lines at the start of a paragraph (Schusterjungen)
\clubpenalty = 10000
% Disable single lines at the end of a paragraph (Hurenkinder)
\widowpenalty = 10000 
\displaywidowpenalty = 10000 % formulas

% Graffiti as in GKP's book "Concrete Mathematics"
% thanks to Lorenzo Pantieri and Enrico Gregorio
\def\graffito@setup{%
   \slshape\footnotesize%
   \parindent=0pt \lineskip=0pt \lineskiplimit=0pt %
   \tolerance=2000 \hyphenpenalty=300 \exhyphenpenalty=300%
   \doublehyphendemerits=100000%
   \finalhyphendemerits=\doublehyphendemerits}
%\DeclareRobustCommand{\graffito}[1]{\marginpar%
% [\graffito@setup\raggedleft\hspace{0pt}{#1}]%
% {\graffito@setup\raggedright\hspace{0pt}{#1}}}
\let\oldmarginpar\marginpar
\renewcommand{\marginpar}[1]{\oldmarginpar%
 [\graffito@setup\raggedleft\hspace{0pt}{#1}]%
 {\graffito@setup\raggedright\hspace{0pt}{#1}}}
              
% headlines
    \clearscrheadings
    \setheadsepline{0pt}
    \renewcommand{\sectionmark}[1]{\markright{\thesection\enspace\textsc{#1}}} 
    \lehead{\mbox{\llap{\small\thepage\kern2em}\headmark\hfil}}
    \rohead{\mbox{\hfil{\headmark}\rlap{\small\kern2em\thepage}}}
    \renewcommand{\headfont}{\small}  
  %    \DeclareRobustCommand{\fixBothHeadlines}[2]{} % <--- ToDo
    % hack to get the content headlines right (thanks, Lorenzo!)
    \def\toc@heading{%
      \chapter*{\contentsname}
      \@mkboth{\textsc{\contentsname}}{\textsc{\contentsname}}}

% section heading styles     
\usepackage{titlesec}
\usepackage{textcase} % for \MakeTextUppercase
\newfont{\chapterNumber}{eurb10 scaled 7000} % chapter number font
%\AtBeginDocument{\addtocontents{toc}{\protect\vspace{-\cftbeforechapskip}}}

\titleformat{\chapter}[display]%
  {\relax}{\vspace*{-3\baselineskip}\makebox[\linewidth][r]{\color{halfgray}\chapterNumber\thechapter}}{0pt}%
  {\raggedright\spacedallcaps}[\normalsize\vspace*{.8\baselineskip}\titlerule]%

% sections
\titleformat{\section}{\relax}{\textsc{\thesection}}{1em}{\scshape}
% subsections
\titleformat{\subsection}{\relax}{\textsc{\MakeTextLowercase{\thesubsection}}}{1em}{\normalsize\itshape}
% subsubsections
\titleformat{\subsubsection}{\relax}{\textsc{\MakeTextLowercase{\thesubsubsection}}}{1em}{\normalsize\itshape}        
% paragraphs
\titleformat{\paragraph}[runin]{\normalfont\normalsize}{\theparagraph}{0pt}{\scshape}    
% descriptionlabels
    \renewcommand{\descriptionlabel}[1]{\hspace*{\labelsep}\textsc{#1}}   % spacedlowsmallcaps textit textsc                  
% spacing
\titlespacing*{\chapter}{0pt}{1\baselineskip}{1.2\baselineskip}
\titlespacing*{\section}{0pt}{1.25\baselineskip}{1\baselineskip} 
\titlespacing*{\subsection}{0pt}{1.25\baselineskip}{1\baselineskip}
\titlespacing*{\paragraph}{0pt}{1\baselineskip}{1\baselineskip}


% toc style
\usepackage[titles]{tocloft}
\renewcommand{\cftchappresnum}{\scshape\MakeTextLowercase}%
\renewcommand{\cftchapfont}{\normalfont}%
\renewcommand{\cftchappagefont}{\normalfont}%
\let\oldchap=\chapter
\renewcommand*{\chapter}{\secdef{\Chap}{\ChapS}}
\newcommand\ChapS[1]{\oldchap*{#1}}%
% \newcommand\Chap[2][]{%
%        \ifpdf\oldchap[\texorpdfstring{\spacedlowsmallcaps{#1}}{#1}]{#2}%
%        \else\oldchap[\spacedlowsmallcaps{#1}]{#2}%
%        \fi%
% }%
\newcommand\Chap[2][]{\oldchap[\textsc{#1}]{#2}}%
%\textsc{foo} {\scshape bar}

\deffootnote{0em}{0em}{\thefootnotemark\hspace*{.5em}}